The concentration $X_i$ of a given isotope $i$ is:
\cite[Eq.~4.7,][]{TS14}
\begin{equation}
X_i(t_\mathrm{age})= X_{i0} \exp{\left( \frac{t_\mathrm{age} \ln2}{T_{i1/2}} \right)}
\end{equation}
where $X_{i0}$ is present-day concentration, $t_\mathrm{age}$ age (time before present), $T_{i1/2}$ half-life.

\begin{equation}
t_\mathrm{age} = t_{i0} - t
\end{equation}
where $t$ the time after the starting time of the protoplanetary system and $t_{i0}$ refers to the time at which the concentration is known or inferred ($X_{i0}$).  For example, $t_{i0}=4.54$ Byrs for isotope concentrations that are measured at present day.  So we can write:

\begin{equation}
X_i(t) = X_{i0} \exp{\left( \frac{(t_{i0}-t) \ln2}{T_{i1/2}} \right)}
\end{equation}

This is useful because the concentration of extant radionuclides are usually defined at present day.  But for extinct radionuclides the present-day abundance is zero so instead we need to be able to define the initial concentration earlier in history.  Often the isotope concentration is defined as:

\begin{equation}
X_i(t=t_{i0})=X_{i0} = x_{i0} C_0
\end{equation}
where $x_{i0}$ is the fractional abundance of the isotope in the naturally occurring element and $C_0$ is the concentration (abundance) of the element, both defined at $t=t_{i0}$.  The rock type under consideration determines $C_0$ at present-day for Earth rocks, meteorites, etc.  The heat production rate for isotope $i$ is thus:
\begin{equation}
H_i(t)=H_i x_{i0} C_0 \exp{\left( \frac{(t_{i0}-t) \ln2}{T_{i1/2}} \right)}
\end{equation}
%%%
\begin{table}[htbp]
\centering
\begin{tabular}{l c c c}
\hline
& T$_{i1/2}$ (Myr) & $x_{i0}$ & $H_i$ (W/kg)\\
\hline
$^{26}$Al & 0.717 & 0 & 0.3583\\
$^{40}$K & 1248 & $1.1668\times10^{-4}$ & $2.8761\times10^{-5}$\\
K & & & $3.4302\times10^{-9}$\\
$^{60}$Fe & 2.62 & 0 & $3.6579\times10^{-2}$\\
$^{232}$Th & 14000 & 1 & $2.6368\times10^{-5}$\\
$^{235}$U & 704 & 0.0072045 & $5.68402\times10^{-4}$\\
$^{238}$U & 4468 & 0.9927955 & $9.4946\times10^{-5}$\\
U & & & $9.8314\times10^{-5}$
\end{tabular}
\caption[Physical constants for radionuclides]{Physical constants for radionuclides.  $x_{i0}$ is the fractional present-day isotopic abundance ($t_{i0}=4.55\times10^9$ yrs).  Reproduced from \cite{RUE17}.}
\label{table:radionuclides}
\end{table}
%%%
The total heating rate is the sum of heating from all isotopes:
\begin{equation}
H(t) = \sum_i H_i(t)
\end{equation}
This equation is the same as \cite[Eq.~4.8][]{TS14}.  Note that \cite{RUE17} computes bulk element power ensuring internal consistency \citep[Sect.~3.7][]{RUE17}.