We have purposefully chosen to express energy conservation in a magma ocean in terms of entropy to strive toward thermo-dynamically self-consistent (or least thermodynamically ``reasonable'') modelling.  Using the RTpress model as lookup data \citep{WB18} ensures this is the case.  However, there is benefit to considering a range of thermodynamic data from different sources to test models.  Typically, this data is expressed in terms of temperature and pressure, and perhaps quantities are only a function of pressure.  Particularly for modelling exoplanets (Super-Earths), the mineral physics data is poorly constrained.  We therefore outline a workflow below for constructing a pseudo self-consistent thermodynamic model that can be used as input to the magma ocean evolution code. \awnote{pseudo self-consistent may or may not be too strong language. It depends on how this pans out.}

\subsection{General outline of approach}
We assume that $\rho(P)$, $\alpha(P)$, $c_p(P)$ are all provided as a function of pressure $P$ for both the solid and melt phase, and $k$ and $\eta$ are constant but potentially different for the melt and solid phase. Furthermore, we are provided with a liquidus and a solidus curve, expressed in terms of temperature.  The modeller must also specify a reference pressure $P_0$, and the enthalpy ($\Delta h$) or entropy of melting ($\Delta S_{\rm fus}$).  The basic workflow is this:
\begin{itemize}
\item Let's consider the liquidus first.  At $P_0$ we determine the temperature of the liquidus, which must correspond to the liquid temperature by definition.  Using this as a reference point, we can compute a reference liquid adiabat by integrating the following as a function of $T$ and $P$ (since material properties are $P$-dependent):
\begin{equation}
\left( \frac{dT}{dP} \right)_S = \frac{\alpha T}{c_p \rho}
\label{eqn:adiabat}
\end{equation}
Let's say that this is our reference liquid adiabat, where $S=S_0$ by definition.
\item We can now map the liquidus from temperature to (relative) entropy space by measuring the departure of the liquidus (positive or negative) from the reference adiabat.
\begin{equation}
dS = \left( \frac{\partial S}{\partial T} \right)_P dT + \left( \frac{\partial S}{\partial P} \right)_T dP
\end{equation}
\begin{equation}
dS = \left( \frac{c_P}{T} \right) dT
\label{eqn:dS}
\end{equation}
\begin{equation}
\Delta S = \int_{T_1}^{T_{2}} \frac{c_P}{T} dT
\label{eqn:int_dS}
\end{equation}
Since we are integrating along a constant pressure path.  Here, $c_P$ is for the liquid phase.  Here, $T_1$ is the temperature along the reference adiabat at a given pressure and $T_2$ is the temperature at the liquidus.  \dbnote{this seems fine when integrating down from above the liquidus.  But if we have to integrate up to the liquidus then is this OK?.  Do we have to worry about $c_P$ for the mixed phase?} \awnote{This is not a problem. We will simply be using the metastable free energy surface to calculate entropy differences, which is well-defined everywhere even if it is not favored in terms of Gibbs free energy}.  This gives us a liquidus curve as a function of entropy.
\item We can repeat the above two steps for the solidus, to obtain the solidus curve expressed as an entropy perturbation from a solid reference adiabat.
\item So far we have a (relative) solidus and a (relative) liquidus, but have not tied together the two in a common (relative) entropy space.  We need to know the entropy difference between the liquidus and solidus.  Crucially, it is known from theory(?) and/or experiment(?) \awnote{Known from theory and verified by limited experiments} that the entropy of melting is \awnote{roughly constant and independent of pressure} near constant over a large pressure range \dbnote{how large?} \awnote{This holds especially true for MgSiO3, which has an entropy of melting that is constant to within ~3\% over the pressure range of the Earth's mantle Stixrude2009}.  Furthermore, it is a relatively well-constrained property.  Hence we bias our model to strongly favour this constraint \awnote{In truth, it is actually the configurational component of the entropy of fusion that is roughly constant, whereas the volumetric component can vary (due to the changing volume of fusion). If we wanted, we could include this effect, but it is also likely small. I can add refs here too.}:
\begin{enumerate}
\item Perhaps this is most sensibly given by the modeller: $\Delta S_{\rm fus}$ at $P_0$.
\item Or the modeler could give us enthalpy $\Delta h$ at $P_0$, from which $\Delta S_{\rm fus} = \Delta h / T_0$ where $T_0$ is a temperature associated with $P_0$.  \dbnote{I expect that $T_0$ is most meaningful (or possibly the least-incorrect) if this is taken as the temperature along the fusion curve, which we define as equidistant between the liquidus and solidus} \awnote{Yes! Use average of liquidus and solidus temperature}.
\item If we knew, or could estimate $c_P$ in the mixed phase region (presumably as a function of $P-T$), then we could integrate Eqn.~\ref{eqn:dS} from the solidus (or liquidus) to the liquidus (or solidus).  Although I don't think this can work, since $C_p$ depends on $\Delta S_{\rm fus}$ (Eqn.~\ref{eqn:heat_capacity_agg}).\awnote{Agreed. I realized after we hung up that I misspoke. The heat capacity of the mixed phase is dominated by the configuration entropy of fusion. The heat capacity contribution of the solid and liquid phases is negligible, and already folded up into the entropy of melting anyway.}
\end{enumerate}
\item For simplicity, let's say the modeller gives us $\Delta S_{\rm fus}$ (i.e., approach 1. above) and additionally we can estimate an appropriate $\Delta T_{\rm fus}$ at $P_0$ from the liquidus and solidus.  We can now map our liquidus and solidus to a common (relative) entropy coordinate, and compute all necessary quantities in the mixed phase region (Eqn.~\ref{eqn:temperature_agg}, \ref{eqn:density_agg}, \ref{eqn:heat_capacity_agg}, \ref{eqn:thermal_exp_agg}, \ref{eqn:adiabatic_temp_grad}).  This is nice, because although we are necessarily making assumptions, we are ultimately collapsing the thermodynamic description to the pseudo one-component formulation presented in section~\ref{sect:thermodynamic}.  For which, importantly, the magma ocean code is geared towards solving.
\item \textbf{Thermodynamic inconsistency (surprise!)}
  The liquidus and solidus are computed independently of each other, using material properties that we provide ($\rho$, $\alpha$. $c_P$, etc.) which are assumed to be functions of pressure only and different for the solid and liquid phase.  Now in the step above, we pin the entropy difference between the liquidus and solidus at the reference pressure $P_0$, and we *hope* / would expect that this entropy difference is relatively constant for all pressure \dbnote{do we have a rough estimate of ``how'' constant?} \awnote{see above. also, note that Dave did not justify this assumption in his papers with Miki =)}.  But there is absolutely no guarantee this will be the case, and in fact some really funky behaviour could occur such as the liquidus decreasing below the solidus, which is clearly nonsensical.  So to enforce some basic level of thermodynamic consistency, we need to ``tweak something'', and what to tweak is not particularly obvious:
\begin{enumerate}
\item One approach is to change the melt and/or solid properties to try and result in an entropy of fusion that seems reasonable (relatively constant).  This will firstly change the slope of the reference adiabat, through Eqn.~\ref{eqn:adiabat}, but secondly change the entropy difference of the curve relative to the adiabat ($\Delta S$) due to $c_P$ (Eqn.~\ref{eqn:int_dS}).  In this approach we are effectively treating the liquidus and solidus, prescribed in temperature space, as fixed (definitive), and changing the material constants to ensure that a reasonable entropy of fusion is predicted.
\item Alternatively, we could lean towards keeping the material properties fixed, and adjusting either the predicted liquidus or solidus (in entropy space), by essentially moving the relative spacing of the liquidus and solidus (i.e., entropy of fusion) as a function of pressure.  Now when we map these entropy curves back to temperature, we will no longer be matching the liquidus and solidus curves in temperature space that we started with.  But in this particular case, we accept this compromise, and hope that the mismatch will not be too large.
\item Of course, another approach is to combine 1 and 2 above.  To do this systematically would be more difficult (i.e., a robust inversion), but to get something roughly thermodynamically consistent and hence reasonable might not be beyond reach.
\item \awnote{What I think we should do to keep it simple is to request that the user give only one of the melting curve bounds, either the liquidus OR solidus. Then they also provide the entropy of melting and temperature change upon melting at a reference pressure. Then the other curve is calculated fully self-consistently. This will prevent them from getting into trouble or assuming implausible melting behavior just to match a solidus/liquidus curve. We can provide tools to adjust the material properties if need be to better match a target liquidus/solidus pair.} 
\end{enumerate}
\item Finally, our magma ocean evolution code requires knowing the temperature as a function of entropy to compute the capacitance on the LHS in front of $dS/dt$.  For $c_P$ a function of pressure only (but different for melt and solid), we can use Eqn.~\ref{eqn:int_dS}, where $T_1$ is the temperature of the reference adiabat ($T_{\rm ad}$) and $T$ an arbitrary temperature.  Integrating between these limits and rearranging gives:
\begin{equation}
T(P) = T_{\rm ad}(P) \exp{ \left( \frac{\Delta S(P)}{c_P(P)} \right) }
\end{equation}
Here, the pressure-dependence is explicitly noted for each of the terms.  But remember that we are evaluating this expression at constant pressure, so the equation is in fact just an analytic expression to determine the temperature at a given (relative) entropy.  There maybe an offset term to include here, depending how the final (relative) entropy space is determined.
\end{itemize}
