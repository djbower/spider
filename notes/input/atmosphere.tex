\fbox{\parbox{\textwidth}{Dimensional and non-dimensional equation for the mass balance and evolution of mass balance of a volatile.}}

%%%%
%%%%
%%%%
\subsection{Volatile Mass Balance}
The mass balance of a given volatile in the interior of a SPIDER model \citep[e.g.,][]{LMC13} is:
\begin{equation}
m_v^s + m_v^l + m_v^g + m_v^e + m_v^o + m_v^r= m_v^t
\end{equation}
where subscript $v$ denotes a particular volatile and superscripts $s$, $l$, $g$, $o$, $t$ indicates the volatile's mass in the solid, liquid (i.e., melt), gas, and surface liquid ocean, as well as the total mass, respectively.  Superscript $e$ represents the reservoir (lost) due to escape, superscript $o$ accounts for ocean formation (this is just a placeholder and is not currently implemented), and superscript $r$ accounts for the amount of the volatile added (or removed) by chemical reactions.  \textbf{Only the solid, liquid, and atmosphere are physical reservoirs, in the sense that the other reservoirs are fictitious to ensure mass conservation.}. We can express these masses as follows:
\begin{equation}
X_v^s M^s + X_v^l M^l + X_v^g M^g + m_v^e + m_v^o + m_v^r = X_v^{\rm init} M^m
\end{equation}
where $X$ are mass fractions of volatile in each phase (solid, liquid, and gas), relative to the respective \myemph{physical} reservoir size of solid, liquid, and gas.  \myemph{In SPIDER we actually work with scaled volume or mass, i.e. the $4 \pi$ prefactor associated with spherical geometry is omitted for all spherical volume and mass quantities (except for output, where the $4 \pi$ is reintroduced).  Nevertheless, the above equation is valid for physical or scaled masses}.  \textbf{Since we will eventually formulate the equation in terms of the mass concentration in the melt, we will also drop the $l$ superscript from $X_v$ for convenience.} The total mass of the mantle:
\begin{equation}
M^m = M^s + M^l
\end{equation}
This is used because it is most sensible to define an initial mass fraction of volatiles relative to the total mantle mass, which is constant, even though the masses of the solid and liquid mantle evolve with time as the magma ocean cools and crystallises.  Note that the initial mass fraction of volatiles as prescribed by the user must obey the mass balance, which means that the initial values will be adjusted to, for example, be consistent with prescribed chemical reactions.  We assume a simple partition coefficient that relates the mass fraction of the volatile in the solid phase to the mass fraction of the volatile in the liquid phase.
\begin{equation}
k_v = \frac{X_v^s}{X_v}
\end{equation}
We should now consider the \myemph{physical} atmospheric mass of a particular volatile.  First, consider the total atmospheric mass as being composed of $q$ species:
\begin{equation}
m_t^g = m_0^g + m_1^g + m_2^g + \dots + m_q^g = \frac{4 \pi R_p^2}{g} P_s
\end{equation}
where $R_p$ is the planetary radius, and $P_s$ is the surface pressure.  Now express in terms of molar mass $\mu$:
\begin{equation}
\bar{\mu} N = \mu_0^g n_0 + \mu_1^g n_1 + \mu_2^g n_2+ \dots + \mu_q^g n_q = \frac{4 \pi R_p^2}{g} P_s
\end{equation}
where $\bar{\mu}$ is the mean molar mass of the atmosphere, $N$ the total number of moles, and $n_q$ is the number of moles of species $q$.  Now divide through:
\begin{equation}
\frac{\mu_0^g}{\bar{\mu}} \frac{n_0}{N} + \frac{\mu_1^g}{\bar{\mu}} \frac{n_1}{N} + \frac{\mu_2^g}{\bar{\mu}} \frac{n_2}{N}+ \dots + \frac{\mu_q^g}{\bar{\mu}} \frac{n_q}{N} = 1
\end{equation}
The definition of partial pressure $p_q$:
\begin{equation}
\frac{n_q}{N} = \frac{p_q}{P_s}
\end{equation}
Note that the partial pressure must vary with optical depth (i.e., height), but for a well-mixed atmosphere the ratio of the partial pressure to the total pressure for a given species is constant.  By reintroducing the constant factors leads to:
\begin{equation}
4 \pi R_p^2 \left( \frac{\mu_0^g}{\bar{\mu}} \right) \frac{p_0}{g} + 4 \pi R_p^2 \left( \frac{\mu_1^g}{\bar{\mu}} \right) \frac{p_1}{g} + \dots + 4 \pi R_p^2 \left( \frac{\mu_q^g}{\bar{\mu}} \right) \frac{p_q}{g} = \frac{4 \pi R_p^2 P_s}{g} = m_{\rm t}^{\rm g}
\end{equation}
Demonstrating that the mass of a given volatile species $q$ is related to the \myemph{surface partial pressure} as:
\begin{equation}
m_q^g = 4 \pi R_p^2 \left( \frac{\mu_q^g}{\bar{\mu}} \right) \frac{p_q}{g}
\end{equation}
Many previous studies that consider outgassing of multiple volatiles species do not use this correct expression \citep[e.g.,][]{ET08,LMC13,SMD17,NKT19}.  This is discussed in \cite{BKW19}.  The \myemph{physical} atmospheric mass of a particular volatile is given by:
\begin{equation}
m_v^g = X_v^g M^g = 4 \pi R_p^2 \left( \frac{\mu_v^g}{\bar{\mu}} \right) \frac{p_v}{g}
\end{equation}
\myemph{We must exclude the factor of $4 \pi$ for scaled mass!}.  $p_v(X_v)$ is the partial pressure of the volatile which is a function of the mass fraction in the liquid phase, i.e., by a modified (power-law form) of Henry's law:
\begin{equation}
p_v ( X_v ) = \left( \frac{X_v}{\alpha_v} \right)^{\beta_v}, \qquad \frac{dp_v}{d X_v} = \frac{\beta_v}{\alpha_v} \left( \frac{X_v}{\alpha_v} \right)^{\beta_v-1}
\label{eq:Henry_mod}
\end{equation}
where $\alpha_v$ and $\beta_v$ are parameters for each volatile.  The ``standard'' Henry's law is recovered when $\beta_v=1$, but allowing a power-law form provides more flexibility for volatiles that do not follow Henry's law exactly ($\beta_v \neq 1$).  \myemph{From now on we will only consider the scaled mass which omits the $4 \pi$ factor associated with spherical geometry}.  Basically, we omit the $4 \pi$ scaling from $M^s$, $M^l$, and $M^m$, and drop the $4 \pi$ in front of the atmosphere term.  Similarly, $4 \pi$ is dropped from the other mass terms: $m_v^e$, $m_v^o$, and $m_v^r$:
\boxedeq{eq:dimvolatile}{X_v (k_v M^s + M^l) + \frac{R_p^2}{g} \left( \frac{\mu_v^g}{\bar{\mu}} \right) p_v + m_v^e + m_v^o + m_v^r = X_v^{\rm init} M^m}
And note, importantly, that we solve for the volatile mass fraction in the liquid phase, from which we can subsequently compute the volatile mass in the solid and gas phase.  The above equation also shows that the volatile abundances are coupled through the mean molecular weight:
\begin{equation}
\bar{\mu} = \sum_v^q \left( \frac{p_v \mu_v}{P_s} \right), \qquad P_s = \sum_v^q p_v
\label{eq:atmosphere_molar_mass}
\end{equation}
Instead, we could choose to represent the volatile mass balance using the partial pressure as the primary (solution) variable, by recognising that:
\begin{equation}
X_v^l = \alpha_v p_v ^{1/\beta}, \qquad \frac{dX_v^l}{d p_v} = \frac{\alpha_v}{\beta_v} \left( \frac{X_v}{\alpha_v} \right)^{1-\beta_v}
\label{eq:Henry_mod2}
\end{equation}
where now a zero Henry coefficient ($\alpha_v=0$) will exclude a dissolved volatile content.  This is a crucial advantage of using partial pressure as the primary solution variable rather than the dissolved volatile content in the melt ($X_v^l$).  Following this approach, the mass balance instead is:
\boxedeq{eq:dimvolatile2}{k_v^l \alpha_v p_v^{1/\beta} (k_v^s M^s + M^l) + \frac{R_p^2}{g} \left( \frac{\mu_v^g}{\bar{\mu}} \right) p_v + m_v^e + m_v^o + m_v^r = X_v^{\rm init} M^m}
In fact, we don't need to follow this alternative derivation to the end.  We can just replace $X_v$ with $p_v$ by substitution.
%%%%
%%%%
%%%%
\subsection{Non-dimensionalisation}
We now sequentially introduce the terms that form the dimensionalisation scheme for the volatile mass balance.
\subsubsection{Mass}
In the code, we non-dimensionalise all masses as:
\begin{equation}
M = \rho_0 R_0^3 \hat{M} = M_0 \hat{M}
\end{equation}
Note that $\rho_0$ and $R_0$ are chosen to ensure that the solution quantities are around unity value.  They do not necessarily correspond to a physically meaningful value.  Therefore, $R_0$ is not necessarily the radius of the planet $R_p$.  $\hat{M}$ is the non-dimensional mass.  The non-dimensional mass balance is:
\begin{equation}
k_v X_v \hat{M}^s + X_v \hat{M}^l + \frac{R_p^2}{M_0 g} \left( \frac{\hat{\mu}_v^g}{\hat{\bar{\mu}}} \right) p_v + \frac{m_v^e}{M_0} + \frac{m_v^o}{M_0} + \frac{m_v^r}{M_0} = X_v^{\rm init} \hat{M}^m
\end{equation}
Note that we have also non-dimensionalised the molar masses in this step, although since we take the ratio of these quantities it probably isn't required.
%%%%
\subsubsection{Volatile Concentration}
It's convenient to express volatile concentration as a scaled version of mass fraction.  This scaling is to ensure we can scale the volatile evolution equations similar to other quantities in the system of equations (like entropy, entropy gradient, etc.):
\begin{equation}
X_v = V_0 \hat{X}_v
\end{equation}
If $V_0=10^{-6}$ then the units of $\hat{X}_v$ are parts-per-million (ppm)\\
if $V_0=10^{-2}$ then the units of $\hat{X}_v$ are weight percent (wt\%)\\
if $V_0=1$ then the units of $\hat{X}_v$ are mass fraction\\~\\
Therefore, the mass balance:
\begin{equation}
k_v \hat{X}_v \hat{M}^s + \hat{X}_v \hat{M}^l + \frac{R_p^2}{V_0 M_0 g} \left( \frac{\hat{\mu}_v^g}{\hat{\bar{\mu}}} \right) p_v + \frac{1}{V_0 M_0} \left( m_v^e + m_v^o + m_v^r \right) = \hat{X}_v^{\rm init} \hat{M}^m
\end{equation}
Collect terms:
\begin{equation}
\hat{X}_v (k_v \hat{M}^s + \hat{M}^l) + \frac{R_p^2}{V_0 M_0 g} \left( \frac{\hat{\mu}_v^g}{\hat{\bar{\mu}}} \right) p_v + \frac{1}{V_0 M_0} \left( m_v^e + m_v^o + m_v^r \right) = \hat{X}_v^{\rm init} \hat{M}^m
\end{equation}
Note that the $X_v$ within $p_v$ has \textbf{not} yet been non-dimensionalised (see below).
%%%%
\subsubsection{Power Law Solubility}
A solubility relation defines the relationship between the dissolved volatile content and the partial pressure.  Here, we derive the scalings for the modified Henry's law.
Convert the partial pressure expression to non-dimensional form.  First, introduce $\hat{X_v}$ and a non-dimensional pressure scale $P_0$:
\begin{equation}
p_v( X_v ) = \left( \frac{X_v}{\alpha_v} \right)^{\beta_v} \implies
 \hat{p}_v( \hat{X}_v ) = \frac{1}{P_0} \left( \frac{V_0 \hat{X}_v}{\alpha_v} \right)^{\beta_v}
 \end{equation}
Therefore:
\begin{equation}
\hat{p}_v ( \hat{X_v} ) = \left( \frac{\hat{X_v}}{P_0^{1/\beta_v} V_0^{-1} \alpha_v} \right)^{\beta_v}
\end{equation}
Now we can define a non-dimensional Henry's constant:
\begin{equation}
\hat{\alpha}_v = \alpha_v \frac{P_0^\frac{1}{\beta_v}}{V_0}
\end{equation}
$P_0$ is a pressure scale that relates to the primary non-dimensional quantities.  In the code, we input $\alpha_v$ in units of ppm/Pa$^{1/\beta}$.  So to \myemph{non-dimensionalise the Henry constant we must also account for the conversion from ppm to mass fraction}:
\begin{equation}
\hat{\alpha}_v = \frac{\alpha_v^{{\rm ppm/Pa}^{1/\beta}}}{10^6} \frac{P_0^\frac{1}{\beta_v}}{V_0}
\label{eq:henryconstnd}
\end{equation}
Therefore, the non-dimensional partial pressure and its derivative are:
\begin{equation}
\hat{p}_v ( \hat{X}_v ) = \left( \frac{\hat{X}_v}{\hat{\alpha}_v} \right) ^ {\beta_v}, \qquad \frac{d \hat{p}_v}{d \hat{X}_v} = \frac{\beta_v}{\hat{\alpha}_v} \left( \frac{\hat{X}_v}{\hat{\alpha}_v} \right)^{\beta_v-1}
\end{equation}
In the latest version of SPIDER, we actually use the inverse functions to enable zero solubilities to be used without introducing numerical problems (since now, the Henry coefficient is in the numerator and will result in zero volatile content in the melt for zero solubility):
\begin{equation}
\hat{X}_v = \hat{\alpha}_v {\hat{p}_v} ^{1/\beta}, \qquad \frac{d \hat{X}_v}{d \hat{p}_v} = \frac{\hat{\alpha}_v}{\beta_v} \left(\frac{\hat{X}_v}{\hat{\alpha}_v}\right)^{1-\beta_v}
\end{equation}
%%%%
%%%%
%%%%
\subsubsection{Sossi Solubility}
% from Paulo Sossi's e-mail 20th August 2020
% H2O (ppm)=534(fH2O)^0.5+723(fH2) (fugacities in bar)
%%%
For H$_2$O:
\begin{equation}
X_v = A{f_{H_2O}}^\frac{1}{2}+B f_{H_2}
\end{equation}
where $A=534$ and $B=723$ when $X_v$ is ppmw, and \textbf{fugacities in bar}.  Using Eq.~\ref{eq:SFequil}, where the ratio of H$_2$ and H$_2$O is constrained by an oxygen fugacity buffer, we can eliminate $f_{H_2}$ using $f_{O_2}$:
\begin{equation}
X_v = A{f_{H_2O}}^\frac{1}{2}+B K \frac{f_{H_2O}}{f_{O_2}^{1/2}}
\end{equation}
Note that $f_{O_2}$ is already non-dimensional according to the total surface pressure (by convention).  In SPIDER, we numerically compute a so-called modified equilibrium constant $G=K/f_{O_2}^{1/2}$ since this avoids introducing numerical noise.  Therefore:
\begin{equation}
X_v = A{f_{H_2O}}^\frac{1}{2}+B G f_{H_2O}
\end{equation}
Comparison with the power law solubility in the previous section reveals that the above is just a linear combination of two power laws with different $\beta$ ($\beta=2$ for the $A$ term and $\beta=1$ for the $B$ term).  Therefore, we should use the same scalings for $A$ and $B$ as determined previously for the Henry constant (Eq.~\ref{eq:henryconstnd}):
\begin{equation}
\hat{A} = \frac{A^{{\rm ppm/Pa}^{1/\beta}}}{10^6} \frac{P_0^\frac{1}{\beta}}{V_0}, \qquad \hat{B} = \frac{B^{{\rm ppm/Pa}^{1/\beta}}}{10^6} \frac{P_0^\frac{1}{\beta}}{V_0}
\end{equation}
Where the non-dimensional solubility law is:
\begin{equation}
\hat{X}_v = \hat{A} {\hat{p}_v}^\frac{1}{2}+ \hat{B} G \hat{p}_v
\end{equation}
Paolo provided $B$ = 723 ppm / bar (of H2) where for the $B$ coefficient $\beta=1$.  Hence, we must convert the pressure scale to Pa:
\begin{equation}
B^{\rm ppm/Pa} = B^{\rm ppm/bar} \times 10^{-5} 
\end{equation}
For $A$ we must account for the fact that $\beta \neq 1$:
\begin{equation}
A^{{\rm ppm/Pa}^{1/\beta}} = A^{{\rm ppm/bar}^{1/\beta}} \times (10^{-5})^{1/\beta}
\end{equation}
So clearly in general:
\begin{equation}
\zeta^{{\rm ppm/Pa}^{1/\beta}} = \zeta^{{\rm ppm/bar}^{1/\beta}} \times (10^{-5})^{1/\beta}
\end{equation}
As before, we also need the derivative for the time stepper:
\begin{equation}
\frac{d \hat{X}_v}{d \hat{p}_v} = \frac{\hat{A}_v}{\beta_v} \left(\frac{\hat{X}_v}{\hat{A}_v}\right)^{1-\beta_v} + G \frac{\hat{B}_v}{\beta_v} \left(\frac{\hat{X}_v}{\hat{B}_v}\right)^{1-\beta_v} + \frac{dG}{d \hat{p}_v} \hat{B} \hat{p}_v
\end{equation}
But in fact, the time-stepper wants the time derivative of abundance change in the melt:
\begin{equation}
\frac{d \hat{X}_v}{d t} = \frac{d \hat{p}_v}{d t} \frac{\hat{A}_v}{\beta_v} \left(\frac{\hat{X}_v}{\hat{A}_v}\right)^{1-\beta_v} + \frac{d \hat{p}_v}{d t} G \frac{\hat{B}_v}{\beta_v} \left(\frac{\hat{X}_v}{\hat{B}_v}\right)^{1-\beta_v} + \frac{dG}{d\hat{T}} \frac{d\hat{T}}{dt} \hat{B} \hat{p}_v
\end{equation}

%%% can remove below once above confirmed to work
\begin{comment}
Introduce dimensional scalings and associate fugacities with partial pressure (ideal gas approximation):
\begin{equation}
V_0 \hat{X}_v = AP_0^{1/2} {p_{H_2O}}^\frac{1}{2}+B G P_0 p_{H_2O}
\end{equation}
Rearranging:
\begin{equation}
\hat{X}_v = \frac{A P_0^{1/2}}{V_0} {p_{H_2O}}^\frac{1}{2}+ \frac{B P_0}{V_0} G p_{H_2O}
\end{equation}
Therefore:
\begin{equation}
\hat{A} = \frac{A P_0^{1/2}}{V_0}, \qquad \hat{B} = \frac{B P_0}{V_0}
\end{equation}
We can generalise the scaling if we introduce some arbitrary pressure exponent, so we can account for both the $A$ and $B$ coefficients:
\begin{equation}
\hat{\zeta} = \frac{\zeta P_0^\beta}{V_0}
\end{equation}
This is comparable to the scaling for the non-dimensional Henry constant $\alpha$ calculated above.  Rearranging:
\begin{equation}
\zeta = \frac{\hat{\zeta} V_0}{P_0^\beta}
\end{equation}
In the code, we input $\zeta$ in units of ppm/Pa$^\beta$ to keep with convention.  
\end{comment}


\subsubsection{TODO: move elsewhere}
Now the mass balance becomes, noting that a factor of $P_0$ appears in the numerator on the penultimate term on the LHS because we introduce the non-dimensional pressure:
\begin{equation}
\hat{X}_v (k_v \hat{M}^s + \hat{M}^l) + \frac{R_p^2 P_0}{V_0 M_0 g} \left( \frac{\hat{\mu}_v^g}{\hat{\bar{\mu}}} \right) \hat{p}_v + \frac{1}{V_0 M_0} \left( m_v^e + m_v^o + m_v^r \right) = \hat{X}_v^{\rm init} \hat{M}^m
\end{equation}
%%%%
Note that the as yet unspecified mass reservoirs ($m_v^e$, $m_v^o$, and $m_v^r$) in the last term on the LHS all exclude a factor of $4\pi$ since we are only considering scaled masses now.
%%%%


\subsubsection{Gravity and Radius}
Gravity is non-dimensionalised in SPIDER as:
\begin{equation}
g = \frac{S_0 T_0}{R_0} \hat{g}
\end{equation}
and the radius of the planet $R_p$ is non-dimensionalised by $R_0$.  Hence the mass balance becomes:
\begin{equation}
\hat{X}_v (k_v \hat{M}^s + \hat{M}^l) + \frac{\hat{R}_p^2}{\hat{g}}\frac{P_0 R_0^3}{V_0 M_0 S_0 T_0} \left( \frac{\hat{\mu}_v^g}{\hat{\bar{\mu}}} \right) \hat{p}_v + \frac{1}{V_0 M_0} \left( m_v^e + m_v^o + m_v^r \right) = \hat{X}_v^{\rm init} \hat{M}^m
\end{equation}
Now the scaling constants in the penultimate term on the LHS are, according to the non-dimensional scheme in SPIDER:
\begin{equation}
\frac{P_0 R_0^3}{M_0 S_0 T_0} = \frac{\rho_0 S_0 T_0 R_0^3}{\rho_0 R_0^3 S_0 T_0} = 1
\end{equation}
%%%%
\subsubsection{Volatile Mass Balance}
Therefore the mass balance in non-dimensional form is, \myemph{where again, all reservoirs are scaled masses without the $4 \pi$ term}:
\boxedeq{}{\hat{X}_v (k_v \hat{M}^s + \hat{M}^l) + \frac{\hat{R}_p^2}{V_0 \hat{g}} \left( \frac{\hat{\mu}_v^g}{\hat{\bar{\mu}}} \right) \hat{p}_v + \frac{1}{V_0 M_0} \left( m_v^e + m_v^o + m_v^r \right) = \hat{X}_v^{\rm init} \hat{M}^m \label{eq:volevo}}
%%%%
\subsubsection{Scaled Non-dimensional Mass}
By inspection of Eq.~\ref{eq:volevo}, \myemph{we can determine the scaled (i.e., without the prefactor of $4\pi$) non-dimensional masses of volatiles}:
\begin{equation}
\hat{m}_v^s = \hat{X}_v k_v \hat{M}^s
\end{equation}
\begin{equation}
\hat{m}_v^l = \hat{X}_v \hat{M}^l
\end{equation}
\begin{equation}
\hat{m}_v^g = \frac{\hat{R}_p^2}{V_0 \hat{g}} \left( \frac{\hat{\mu}_v^g}{\hat{\bar{\mu}}} \right) \hat{p}_v
\end{equation}
The remaining mass reservoirs (superscript $e$, $o$, and $r$) all follow this scaling (taking $m_v^e$ as an example, and noting that these quantities also omit the $4 \pi$ factor).
\begin{equation}
\hat{m}_v^e = \frac{m_v^e}{V_0 M_0}
\end{equation}
%%%%
\subsubsection{Dimensional Physical Mass}
We provide a dimensional scaling for each parameter to give a meaningful physical output for plotting and analysis.  For the \myemph{physical mantle reservoir masses of melt and solid phases (note $4 \pi$ term), where the additional superscript $p$ denotes physical}:
\begin{equation}
M^{\rm sp} = \hat{M}^{\rm s} \cdot 4 \pi M_0, \qquad M^{\rm lp} = \hat{M}^{\rm l} \cdot 4 \pi M_0
\end{equation}
For each volatile, we compute the dimensional mass for all reservoirs using the same scaling (taking $m_v^{sp}$ as an example):
\begin{equation}
m_v^{sp} = V_0 \cdot \hat{m}_v^s \cdot 4 \pi M_0
\end{equation}
%%%%
%%%%
%%%%
\subsection{Initial Volatile Concentration}
For an initial condition it is desirable to prescribe $\hat{X}_v^{\rm init}$, but we must then compute $\hat{p_v}$ according to the mass balance since this is the quantity that is actually solved for (i.e., evolved with time).  We also need to realise that our ``desired'' initial volatile concentration may be incompatible with the chemical equilibrium criteria that we (optionally) prescribe, and therefore some mass may be necessarily exchanged through chemical reactions.  Therefore, to determine the initial partial pressure we solve Eq.~\ref{eq:volevo} for a given thermal state of the interior, i.e. the masses of solid and liquid are known.

For the case of no chemical reactions, $\hat{m}_v^r=0$ for all volatiles and hence the only coupling between volatiles is through their partial pressures, which all contribute to the total surface pressure.  For chemical reactions, we can solve for the initial volatile abundance that obeys the prescribed chemical equilibrium laws by rearranging:
\begin{equation}
\frac{\hat{X}_v}{\hat{M}^m} (k_v \hat{M}^s + \hat{M}^l) + \frac{1}{V_0 \hat{M}^m} \frac{\hat{R}_p^2}{\hat{g}} \left( \frac{\hat{\mu}_v^g}{\hat{\bar{\mu}}} \right)\hat{p}_v = \hat{X}_v^{\rm init} - \frac{\hat{m}_v^r}{\hat{M}^m} \equiv \hat{X}_v^{\rm new\ init} 
\label{eq:volinit3}
\end{equation}
Effectively, this initialises the mass reaction term $\hat{m}^r_v$ to zero at the start of the model run, and computes both the new total initial volatile concentration $\hat{X}_v^{\rm new\ init}$ as well as the initial volatile abundance in the melt $\hat{X}_v$.  
%%%%
%%%%
%%%%
\subsection{Chemical Reactions}
\label{sect:chemreact}
\subsubsection{Simple example: $[\rm{H}_2O]\leftrightarrow \frac{1}{2} [\rm{O}_2] + [\rm{H}_2]$}
For simplicity, again consider a single chemical reaction that transfers mass between two volatile species, taking the following as an example:
\begin{equation}
    [\rm{H}_2O]\leftrightarrow \frac{1}{2} [\rm{O}_2] + [\rm{H}_2]
    \label{eq:reaction}
\end{equation}
Eq.~\ref{eq:reaction} is balanced in terms of molar concentration (square brackets), and the equilibrium constant is defined as \cite[e.g.][]{SF17}:
\begin{equation}
K=\frac{p_{\rm H_2} f_{\rm O_2}^{1/2}}{p_{\rm H_2\rm O}}
\label{eq:SFequil}
\end{equation}
Although chemical reactions are naturally expressed in terms of moles (or molecules), we need a formulation that considers mass, since ultimately we are tracking mass addition to, and removal from, interior and exterior reservoirs.  Mass is conserved:
\begin{equation}
m_{H_2O} = m_{O2} + m_{H_2} \implies X_{H_2O} \mu_{H_2O} = X_{O_2} \mu_{O_2} + X_{H_2} \mu_{H_2}
\label{eq:reaction_mass}
\end{equation}
Where $X_y$ denotes the number of moles of a particular species.  Normalise by the mass of the H$_2$O reservoir:
%\begin{equation}
%m_{H_2O} = \frac{X_{tot}}{P_s} p_{H_2O} \mu_{H_2O}
%\end{equation}
\begin{equation}
m_{O_2} = \left( \frac{X_{O_2}}{X_{H_2O}} \right) \left(\frac{\mu_{O_2}}{\mu_{H_2O}}\right) m_{H_2O} = \frac{1}{2} \left(\frac{\mu_{O_2}}{\mu_{H_2O}}\right) m_{H_2O}
\end{equation}
\begin{equation}
m_{H_2} = \left( \frac{X_{H_2}}{X_{H_2O}} \right) \left(\frac{\mu_{H_2}}{\mu_{H_2O}}\right) m_{H_2O} = \left(\frac{\mu_{H_2}}{\mu_{H_2O}}\right) m_{H_2O}
\end{equation}
So in short, we have one extra equation (the equilibrium expression), which can be used to solve for one extra quantity (the mass exchange of H$_2$O), from which we can recover the mass exchange for the other species using the stoichiometry of the reaction and mass balance.
%%%%
%%%%
\subsubsection{General example}
\label{sec:general_reaction}
\begin{equation}
    a[\rm{A}] + b[\rm{B}] \leftrightarrow c[\rm{C}] + d[\rm{D}]
    \label{eq:reaction_general}
\end{equation}
Equilibrium constant:
\begin{equation}
K = \frac{C^c D^d}{A^a B^b}
\label{eq:equilibrium_constant}
\end{equation}
\begin{equation}
\frac{m_{A}}{m_{A}} = 1
\end{equation}
\begin{equation}
\frac{m_{B}}{m_{A}} = \frac{b \mu_{B}}{a \mu_{A} }
\end{equation}
\begin{equation}
\frac{m_{C}}{m_{A}} = \frac{c \mu_{C}}{a \mu_{A} }
\end{equation}
\begin{equation}
\frac{m_{D}}{m_{A}} = \frac{d \mu_{D}}{a \mu_{A} }
\end{equation}
where again we can use Eq.~\ref{eq:equilibrium_constant} as the extra equation, and then solve for $m_A$.  Also, the time derivatives are trivial, since the stoichiometry coefficients and molar masses are constants.  Hence, for example:
\begin{equation}
\frac{\partial m_B}{\partial t} = \frac{b \mu_{B}}{a \mu_{A} } \frac{\partial m_A}{\partial t}
\end{equation}
%%%%
%%%%
\subsection{Evolution Equation}
Remember that we compute a RHS in SPIDER that represents the time-derivative (update) of a solution quantity.  We are going to solve the volatile evolution equations within the system of equations that include the update to entropy, etc.  \textbf{Strictly speaking, it is not necessary to integrate to find the equilibrium volatile abundance in the melt in the simplest situations.  You could save computations (and accumulated error) by instead solving Eq.~\ref{eq:volevo}, but this would not allow you to introduce time-dependent atmospheric escape for example.}  Therefore, we define a non-dimensional time (which is given by the existing parameters in the system by dimensional analysis) as:
\begin{equation}
t = \tau_0 \hat{t}
\end{equation}
And taking the time derivative of the volatile mass balance equation:
\begin{align}
& \frac{d \hat{X}_v}{d \hat{t}} (k_v \hat{M}^s + \hat{M}^l) +
\hat{X}_v \left( k_v \frac{d \hat{M}^s}{d \hat{t}} + \frac{d \hat{M}^l}{d \hat{t}} \right) + 
\frac{\hat{R}_p^2 \hat{\mu}_v^g}{V_0 \hat{g}} \frac{d}{d\hat{t}} \left( \frac{\hat{p}_v}{\hat{\bar{\mu}}} \right)\\
&\quad + \frac{\tau_0}{V_0 M_0} \left( \frac{dm_v^e}{dt} +  \frac{dm_v^o}{dt} + \frac{dm_v^r}{dt} \right) = 0
\end{align}
Since the final term on the LHS has not been defined, we keep the terms dimensional inside the brackets.  This is because we want to define some quantities in terms of SI units, such as escape rates (kg/s).  Note that here the terms such as escape $m_v^e$ are still omitting the $4 \pi$ factor.
\subsubsection{Mean molar mass of atmosphere}
The mean molar mass $\hat{\bar{\mu}}$ of the atmosphere evolves during outgassing and therefore is a time-dependent quantity.  By the product rule:
\begin{equation}
\frac{d}{d\hat{t}} \left( \frac{\hat{p}_v}{\hat{\bar{\mu}}} \right) = \hat{p}_v \frac{d}{d\hat{t}} \left( \frac{1}{\hat{\bar{\mu}}} \right) + \frac{1}{\hat{\bar{\mu}}} \frac{d}{d\hat{t}} \left( \hat{p}_v \right)
\label{eq:atmos_evo}
\end{equation}
The second term can be calculated using the chain rule:
\begin{equation}
\frac{1}{\hat{\bar{\mu}}} \frac{d}{d\hat{t}} \left( \hat{p}_v \right) = \frac{1}{\hat{\bar{\mu}}} \left( \frac{d \hat{p}_v}{d \hat{X}_v} \frac{d \hat{X}_v}{d\hat{t}} \right)
\end{equation}
%%% two species only %%%
\subsubsection{Two volatile species}
To deal with the first term, recall that for two species (Eq.~\ref{eq:atmosphere_molar_mass}):
\begin{equation}
\hat{\mu}_{\rm t} = \left( \frac{\hat{p}_C}{\hat{p}_C+\hat{p}_H} \right) \hat{\mu}_C + \left( \frac{\hat{p}_H}{\hat{p}_C+\hat{p}_H} \right) \hat{\mu}_H
\end{equation}1
where subscript $C$ denotes CO$_2$ and $H$ denotes H$_2$O, and $\hat{p}$ is partial pressure and $\hat{\mu}$ is molar mass.  Taking the time derivative of $\hat{\mu}_{\rm t}$ noting that only the $\mu$'s are independent of time:
\begin{equation}
\frac{d}{d\hat{t}} \left( \frac{1}{\hat{\mu}_{\rm t}} \right) = -\hat{\mu}_{\rm t}^{-2} \frac{d}{d\hat{t}} \hat{\mu}_{\rm t}
\end{equation}
\begin{equation}
\frac{d}{d\hat{t}} \hat{\mu}_{\rm t} = \frac{d}{d\hat{t}} \left[ \left( \frac{\hat{p}_C}{\hat{p}_C+\hat{p}_H} \right) \hat{\mu}_C + \left( \frac{\hat{p}_H}{\hat{p}_C+\hat{p}_H} \right) \hat{\mu}_H \right]
\end{equation}
This is tedious but straightforward to compute (and Mathematica helps!):
\begin{align}
\frac{d}{d\hat{t}} \hat{\mu}_{\rm t} &= \left( \frac{\hat{\mu}_C}{\hat{p}_C+\hat{p}_H} \right) \frac{d\hat{p}_C}{d\hat{t}} - \frac{\hat{\mu}_C \hat{p}_C}{(\hat{p}_C+\hat{p}_H)^2} \left( \frac{d\hat{p}_C}{d\hat{t}}+\frac{d\hat{p}_H}{d\hat{t}} \right)\\
&+ \left( \frac{\hat{\mu}_H}{\hat{p}_C+\hat{p}_H} \right) \frac{d\hat{p}_H}{d\hat{t}} - \frac{\hat{\mu}_H \hat{p}_H}{(\hat{p}_C+\hat{p}_H)^2} \left( \frac{d\hat{p}_C}{d\hat{t}}+\frac{d\hat{p}_H}{d\hat{t}} \right)
\end{align}
Simplifying:
\begin{equation}
\frac{d}{d\hat{t}} \hat{\mu}_{\rm t} = \frac{(\hat{\mu}_C-\hat{\mu}_H)}{(\hat{p}_C+\hat{p}_H)^2} \left( \hat{p}_H \frac{d\hat{p}_C}{d\hat{t}} - \hat{p}_C \frac{d\hat{p}_H}{d\hat{t}} \right)
\end{equation}
Putting it all together (i.e., returning to Eq.~\ref{eq:atmos_evo}):
\begin{equation}
\frac{d}{d\hat{t}} \left( \frac{\hat{p}(\hat{X}^{\rm l})}{\hat{\mu}_{\rm t}} \right) = -\frac{\hat{p}(\hat{X}^{\rm l})}{\hat{\mu}_{\rm t}^2} \frac{(\hat{\mu}_C-\hat{\mu}_H)}{(\hat{p}_C+\hat{p}_H)^2} \left( \hat{p}_H \frac{d\hat{p}_C}{d\hat{t}} - \hat{p}_C \frac{d\hat{p}_H}{d\hat{t}} \right) + \frac{1}{\hat{\mu}_{\rm t}} \left( \frac{d \hat{p}}{d \hat{X}^{\rm l}} \frac{d \hat{X}^{\rm l}}{d\hat{t}} \right)
\end{equation}
To recap, we are considering the concentration of a given volatile $\hat{X}^{\rm l}$, which can either be CO$_2$ or H$_2$O, and we need to know the time derivative of this quantity.  \myemph{Through the mean molar mass of the atmosphere, the volatiles are now coupled, and we can no longer determine the time derivative of either volatile independently of the other.  Rather we must solve a coupled system of equations at each time step to give us the time update of all the volatiles under consideration.}.  The above equations may be expressed slightly differently, but should be identical to the equations in \citet[Appendix A,][]{BKW19}.
%%%%
\subsubsection{$n$ volatile species}
For $n$ species, where $(t)$ is used to emphasise the time-dependent quantities, we just need to derive a general expression for the rate-of-change of the mean molar mass of the atmosphere:
\begin{equation}
\hat{\mu}_{\rm t}(t) = \sum_i^n \left( \frac{\hat{p}_i(t) \hat{\mu}_i}{\sum_j^n \hat{p}_j(t)} \right)
\end{equation}
Time derivative:
\begin{equation}
\frac{d \hat{\mu}_{\rm t}}{d t} = \sum_i^n \hat{\mu}_i \frac{d}{d t} \left( \frac{\hat{p}_i(t)}{\sum_j^n \hat{p}_j(t)} \right)
\end{equation}
Note that the term within the time derivative is the volume mixing ratio of each volatile species.  Break apart the derivative using the product rule:
\begin{equation}
\frac{d \hat{\mu}_{\rm t}}{d t} = \sum_i^n \hat{\mu}_i \left[ \hat{p}_i \frac{d}{d t} \left( \frac{1}{\sum_j^n \hat{p}_j} \right) + \frac{1}{\sum_j^n \hat{p}_j} \frac{d \hat{p}_i}{d t} \right]
\end{equation}
Evaluate:
\begin{equation}
\frac{d \hat{\mu}_{\rm t}}{d t} = \sum_i^n \hat{\mu}_i \left[ \frac{-\hat{p}_i}{\left( \sum_j^n \hat{p}_j \right)^2} \sum_j^n \frac{d \hat{p}_j}{d t} + \frac{1}{\sum_j^n \hat{p}_j} \frac{d \hat{p}_i}{d t} \right]
\end{equation}
Looks simpler when you substitute in the total surface pressure $P_s$ and break apart some terms to show the symmetry (and confirm that the units are correct):
\begin{equation}
\frac{d \hat{\mu}_{\rm t}}{d t} = \sum_i^n \hat{\mu}_i \left[ \frac{1}{P_s} \frac{d \hat{p}_i}{d t} - \frac{\hat{p}_i}{P_s} \frac{1}{P_s} \frac{d P_s}{d t} \right]
\end{equation}
Putting it all together:
\begin{equation}
\frac{d}{dt} \left( \frac{\hat{p}(\hat{X}^{\rm l})}{\hat{\mu}_{\rm t}} \right) = -\frac{\hat{p}(\hat{X}^{\rm l})}{\hat{\mu}_{\rm t}^2} \sum_i^n \hat{\mu}_i \left[ \frac{1}{P_s} \frac{d \hat{p}_i}{d t} - \frac{\hat{p}_i}{P_s} \frac{1}{P_s} \frac{d P_s}{d t} \right] + \frac{1}{\hat{\mu}_{\rm t}} \left( \frac{d \hat{p}}{d \hat{X}^{\rm l}} \frac{d \hat{X}^{\rm l}}{d t} \right)
\end{equation}
The above equations may be expressed slightly differently, but should be identical to the equations in \citet[Appendix A,][]{BKW19}.
%%%%
\subsubsection{Updated evolution equation}
Recall that:
\begin{equation}
\hat{M}^{\rm s} + \hat{M}^{\rm l} = \hat{M}^{\rm m} \equiv \text{constant}
\label{eq:mantle_mass}
\end{equation}
Therefore, we can now just express in terms of the total mantle mass and the mass of liquid (melt) and its time derivative.  \myemph{The evolution equation is, therefore:} 
\begin{align}
&\frac{d \hat{X}_v}{d\hat{t}} \left(k_v \hat{M}^m + (1-k_v) \hat{M}^l \right)
+ \hat{X}_v (1-k_v) \frac{d \hat{M}^l}{d\hat{t}} + \nonumber \\
&\quad \frac{\hat{R}_{\rm p}^2 \hat{\mu}_v^g}{V_0 \hat{g}}
\left[
 \frac{1}{\hat{\bar{\mu}}} \left( \frac{d \hat{p}_v}{d \hat{X}_v} \frac{d \hat{X}_v}{d \hat{t}} \right)
-\frac{\hat{p}_v}{\hat{\bar{\mu}}^2} \sum_i^n \frac{\hat{\mu}_i}{P_s} \left[ \frac{\partial \hat{p}_i}{\partial \hat{t}} - \frac{\hat{p}_i}{P_s} \frac{\partial P_s}{\partial \hat{t}} \right]
\right]+ \nonumber \\
& \qquad \frac{\tau_0}{V_0 M_0} \left( \frac{dm_v^e}{dt} +  \frac{dm_v^o}{dt} + \frac{dm_v^r}{dt} \right) = 0
\end{align}
This cannot be separated into the form of the time derivative equal to a RHS, so the time derivative must instead be numerically calculated within the time stepper.
\myemph{We can also express the evolution equation using partial pressure as the primary (solution) variable, in which case:}
%%%
%%% PRIMARY PV HERE
%%%
\begin{align}
& \frac{d \hat{X}_v}{d\hat{p}_v} \frac{d \hat{p}_v}{d\hat{t}} \left(k_v \hat{M}^m + (1-k_v) \hat{M}^l \right)
+ \hat{X}_v (1-k_v) \frac{d \hat{M}^l}{d\hat{t}} + \nonumber \\
&\quad \frac{\hat{R}_p^2 \hat{\mu}_v^g}{V_0 \hat{g}}
\left[
\frac{1}{\hat{\bar{\mu}}} \left( \frac{d \hat{p}_v}{d \hat{t}} \right)
-\frac{\hat{p_v}}{\hat{\bar{\mu}}^2} \sum_i^n \frac{\hat{\mu}_i}{P_s} \left[ \frac{\partial \hat{p}_i}{\partial \hat{t}} - \frac{\hat{p}_i}{P_s} \frac{\partial P_s}{\partial \hat{t}} \right]
\right]+ \nonumber \\
& \qquad \frac{\tau_0}{V_0 M_0} \left( \frac{dm_v^e}{dt} +  \frac{dm_v^o}{dt} + \frac{dm_v^r}{dt} \right) = 0
\end{align}
Basically, any prescribed relationship between $X_v$ and $p_v$ can be used in this equation, whether it's Henry's law or a more sophisticated formulation.  This offers opportunity to investigate ``fugacity crisis'' scenarios \citep{KFS19,KFS20}
%%%%
\subsubsection{Atmospheric escape formulation}
\textbf{Jeans and constant}\\~\\
The expression for escape due to surface heating is \citep[e.g.,][]{JOY15} is as follows.  Note that the source term already relates to \textbf{scaled mass} which excludes the $4 \pi$ contribution, but this is not the case for $\Phi$
\begin{equation}
\frac{d m_v^e}{dt} = \left( \frac{d m_{\rm v}^{\rm g}}{dt} \right) \mathcal{R} (1 + \lambda_s) \exp(-\lambda_s) + \frac{\Phi}{4 \pi}
\end{equation}
The first term relates to Jean's escape and $\Phi$ enables us to assign an escape rate (a non-Jean's escape contribution).  $\mathcal{R}$ is a fitting parameter based on molecular kinetic simulations of N$_2$ \citep{VJT11,VTE11} and $\lambda_s$ is the surface value of the Jean's parameter $\lambda_s$.  For us, $\lambda_s$ and $\mathcal{R}$ may just be treated as a constant scaling for simplicity.  If $\mathcal{R}=0$ then clearly the Jean's escape is by definition zero (not included).  From dimensional analysis we can see that the terms following the mass derivative must be non-dimensional constants.
\begin{equation}
\lambda_s =  \frac{G M_p^{\rm p} \mu_{\rm v}}{R_p k_b T_s N_A} = \frac{g R_p \mu_{\rm v}}{k_b T_s N_A}
\end{equation}
where $G$ is the gravitational constant, $g$ surface gravity, $M_p$ mass of the planet (superscript p denotes physical mass), $\mu_v$ the molar mass of the volatile, $N_A$ Avogadro's number (units per mol), $k_b$ Boltzmann constant, and $T_s$ surface temperature.  Remember that we deal with scaled masses in SPIDER, but since the Jeans parameter is a physical quantity, we must introduce the correct physical scaling.  Now in fact, the physical $g$ is treated as a constant input parameter by SPIDER and therefore absorbs the factor of $4 \pi$ that is included in the planetary mass.  Non-dimensionalising:
\begin{equation}
\lambda_s = \left( \frac{S_0 T_0 R_0 M_0}{R_0 S_0 \rho_0 R_0^3 T_0} \right) \frac{\hat{g} \hat{R}_p \hat{\mu}_{\rm v}}{\hat{k}_b \hat{T}_s N_A} = \frac{\hat{g} \hat{R}_p \hat{\mu}_{\rm v}^{\rm g}}{\hat{k}_b \hat{T}_s N_A}
\end{equation}
We already computed the growth rate of the atmosphere, i.e. the source rate, and therefore:
\begin{align}
\frac{d \hat{m}_{\rm v}^{\rm esc}}{d\hat{t}} &= \frac{\hat{R}_{\rm p}^2 \hat{\mu}_{\rm v}^{\rm g}}{V_0\hat{g}}
\left[
-\frac{\hat{p}(\hat{X}^{\rm l})}{\hat{\mu}_{\rm t}^2} \sum_i^n \hat{\mu}_i \left[ \frac{1}{P_s} \frac{d \hat{p}_i}{d \hat{t}} - \frac{\hat{p}_i}{P_s} \frac{1}{P_s} \frac{d P_s}{d \hat{t}} \right] + \frac{1}{\hat{\mu}_{\rm t}} \left( \frac{d \hat{p}}{d \hat{X}^{\rm l}} \frac{d \hat{X}^{\rm l}}{d \hat{t}} \right)
\right]\\
& \times \mathcal{R} (1 + \lambda_s) \exp(-\lambda_s) + \frac{\tau_0}{V_0 M_0} \frac{\Phi}{4 \pi}
\end{align}
It is now obvious that we can scale the source time in the evolution equation by a factor to account for atmospheric escape due to Jean's escape:
\begin{equation}
\mathcal{F}_e = 1+\mathcal{R} (1 + \lambda_s) \exp(-\lambda_s)
\end{equation}
But we must additionally include the second term with $\Phi$ in the evolution equation (separate to $\mathcal{F}_e$ above) to enable a non-Jean's escape contribution.

%%%
% Zahnle model
~\\ \textbf{\cite{ZGC19}}\\~\\
For H$_2$, an escape rate accounting for diffusion limited escape and energy limited escape is given by \citep[Eq. 3,][]{ZGC19}:
\begin{equation}
\phi_{H_2} \approx \Gamma \frac{(1 \times 10^{12}) f_{H_2} S}{\sqrt{1+0.006S^2}} \qquad \rm{molecules}\ {\rm cm}^{-2}\ {\rm s}^{-1}
\end{equation}
\begin{equation}
\phi_{H_2} \approx \Gamma \frac{(1 \times 10^{12}) f_{H_2} S}{N_A \sqrt{1+0.006S^2}} \qquad \rm{moles}\ {\rm cm}^{-2}\ {\rm s}^{-1}
\end{equation}
where $N_A$ is Advogadro's number.
\begin{equation}
\phi_{H_2} \approx \Gamma \frac{(10^{16}) f_{H_2} S}{N_A \sqrt{1+0.006S^2}} \qquad \rm{moles}\ {\rm m}^{-2}\ {\rm s}^{-1}
\end{equation}
See the aforementioned paper for details, but $S$ is non-dimensional and $f_{H_2}$ is non-dimensional, and $\Gamma$ is a non-dimensional constant I introduced to scale the whole expression.  Hence all the dimensionality is wrapped up in the constant:
\begin{equation}
\frac{10^{16}}{N_A} \qquad \rm{moles}\ {\rm m}^{-2}\ {\rm s}^{-1}
\end{equation}
The physical escape rate of mass (using the definition of $\phi_{H_2}$ with units of moles and metres), is:
\begin{equation}
\frac{dm_{\rm v}^{\rm esc}}{dt} = 4\pi R_p^2 \phi_{H_2} \mu_{H_2}
\end{equation}
Now non-dimensionalise using the scheme of SPIDER:
\begin{equation}
\frac{d\hat{m}_{\rm v}^{\rm esc}}{dt} = \frac{\tau_0}{V_0 M_0} \frac{4\pi}{4\pi} R_p^2 \phi_{H_2} \mu_{H_2}
\end{equation}
\begin{equation}
\frac{d\hat{m}_{\rm v}^{\rm esc}}{dt} = \frac{\tau_0}{V_0} R_0^2 \hat{R}_p^2 \phi_{H_2} \hat{\mu}_{H_2}
\end{equation}
Note in SPIDER $\hat{\mu}$ still has the units of moles included (but non-dimensional according to mass (kg) due to $M_0$, as done above).  Finally, we can wrap up the dimensional terms within a dimensionless $\hat{\phi}_{H_2}$
\begin{equation}
\hat{\phi}_{H_2} \approx \Gamma \frac{R_0^2 \tau_0 (10^{16}) f_{H_2} S}{V_0 N_A \sqrt{1+0.006S^2}} \qquad \rm{moles}
\end{equation}
Recall that non-dimensional $\mu$ in SPIDER has units of mol$^{-1}$ which will cancel with the moles above.  Hence the non-dimensional escape rate for the \cite{ZGC19} model is:
\begin{equation}
\frac{d\hat{m}_{\rm v}^{\rm esc}}{d\hat{t}} = \hat{R}_p^2 \hat{\phi}_{H_2} \hat{\mu}_{H_2}
\end{equation}
%%%%
%%%%
%%%%
\subsubsection{Chemical reactions}
\paragraph{Overview}
As with the initial condition, we need to prescribe an extra condition when we have reactions based on the reaction quotient (one per reaction).  For the IC we use:
\begin{equation}
Q_p - K Q_r = 0
\end{equation}
Therefore we can take the time derivative, and keep the same scalings since these are preferred by the solver:
\begin{equation}
\frac{\partial Q_p}{\partial t} - K \frac{\partial Q_r}{\partial t} - Q_r \frac{\partial K}{\partial t} = 0
\end{equation}
\paragraph{Reaction quotient: products}
Following the general example for a reaction (Sect.~\ref{sec:general_reaction}), the numerator of the reaction quotient $Q_p$ is:
\begin{equation}
Q_p = C^c D^d
\end{equation}
Note that in the code we use positive stoichiometry coefficients to denote products and negative coefficients to denote reactants.  \textbf{Also, quantities $C$ and $D$ are the volume mixing ratios of the products i.e. $p_v/P_s$, since we wish to retain the total pressure scale for numerical reasons (stability of solver)}.  Now differentiate:
\begin{equation}
\frac{\partial Q_p}{\partial t} = D^d c C^{(c-1)} \frac{\partial C}{\partial t} + C^c d D^{(d-1)} \frac{\partial D}{\partial t}
\end{equation}
Now express $C$ and $D$ in terms of their partial pressure, by recognising that for $C$ (and similarly for $D$):
\begin{equation}
\frac{\partial C}{\partial t} = \frac{\partial}{\partial t} \left( \frac{p_C}{P_s} \right) = \frac{1}{P_s} \frac{\partial p_C}{\partial t} - \frac{p_C}{P_s^2} \frac{\partial P_s}{\partial t}
\end{equation}
\paragraph{Oxygen fugacity}
For several reactions that we consider, the $D$ product is actually oxygen, which is constrained by the evolution of the interior.  Hence we have expressions that give $fO_2$ as a function of temperature $T$.  \textbf{Note that $fO_2$ here is the volume mixing ratio, already normalised by the total (surface) pressure!}.
\begin{equation}
\log_{10} fO_2 = g(T)
\end{equation}
For the form given in \cite{SF17} (and yes, same symbols for constants are used as the stoichiometry above, but in fact they are different!):
\begin{equation}
g_1(T) = a + b \frac{10^3}{T} + c \frac{10^6}{T^2} + d \frac{10^9}{T^3} + f \frac{10^{12}}{T^4}
\end{equation}
For \cite{OS19}:
\begin{equation}
g_2(T) = a T^b
\end{equation}
Oxygen fugacity is:
\begin{equation}
fO_2 = 10^{g(T)}
\end{equation}
Now compute the derivative with respect to temperature $T$, which is needed as part of the chain rule to eventually express in terms of time $t$:
\begin{equation}
\frac{d fO_2}{d T} = \ln(10) \times 10^{g(T)} \frac{dg}{dT}
\end{equation}
Where for the oxygen fugacity models of \cite{SF17}:
\begin{equation}
\frac{dg}{dT} = \frac{dg_1}{dT} = -b \frac{10^3}{T^2} - 2c \frac{10^6}{T^3} - 3d \frac{10^9}{T^4} - 4f \frac{10^{12}}{T^5}
\end{equation}
And for the oxygen fugacity models of \cite{OS19}:
\begin{equation}
\frac{dg}{dT} = \frac{dg_2}{dT} = abT^{(b-1)}
\end{equation}
\paragraph{Reaction quotient: reactants}
Now consider the denominator of the reaction quotient:
\begin{equation}
Q_r = A^{-a} B^{-b} P_s^{-f}
\end{equation}
The above deserves some clarification.  Firstly, in the code we use negative stoichiometry to denote reactants, and hence we negate the coefficients $a$ and $b$ in the above equation.  Secondly, because we are effectively using volume mixing ratios, an extra scaling of $P_s$ appears to ensure that the reaction quotient is non-dimensional.  In the above, I am assuming that the scaling appears in the denominator with an exponent of $f$, and hence again this is also negated in the above equation.  Taking the derivative follows the same pattern as before with the reaction quotient numerator, but now we are considering the reaction quotient denominator:
\begin{equation}
\frac{\partial Q_r}{\partial t} = A^{-a} \left( B^{-b}(-f)P_s^{(-f-1)} \frac{\partial P_s}{\partial t} + P_s^{-f} (-b)B^{(-b-1)} \frac{\partial B}{\partial t} \right) + B^{-b} P_s^{-f} (-a) A^{(-a-1)} \frac{\partial A}{\partial t}
\end{equation}
That is:
\begin{equation}
\frac{\partial Q_r}{\partial t} = A^{-a} B^{-b}(-f)P_s^{(-f-1)} \frac{\partial P_s}{\partial t} + A^{-a} P_s^{-f} (-b)B^{(-b-1)} \frac{\partial B}{\partial t} + B^{-b} P_s^{-f} (-a) A^{(-a-1)} \frac{\partial A}{\partial t}
\end{equation}
For several of the reactions that we consider, there is just one reactant (e.g., H$_2$O) and therefore the second reactant drops out ($B=1$).  Also, often the extra factor of $P_s$ also drops out since $f=0$.  Hence although this equation is general, it is reduced in complexity for several of the reactions that we consider in practice.
\paragraph{Equilibrium constant}
All equilibrium constants $K$ are expressed using the form in \cite{SF17}:
\begin{equation}
\log_{10} K = \frac{a}{T} + b
\end{equation}
where again $a$ and $b$ are constants unrelated to previous usages of these symbols.  We can differentiate with respect to $T$, since by the chain rule this is required if we eventually want $\partial K/\partial T$.  The differential is:
\begin{equation}
\frac{\partial K}{\partial T} = \ln(10) \times 10^{(a/T+b)} \times \frac{-a}{T^2}
\end{equation}
\paragraph{Derivatives with respect to time $t$}
Both fO$_2$ and $K$ a functions of temperature, which is why we have computed derivatives with respect to temperature $T$, since:
\begin{equation}
\frac{\partial fO_2}{\partial t} = \frac{\partial fO_2}{\partial T} \frac{\partial T}{\partial t}
\end{equation}
And similarly:
\begin{equation}
\frac{\partial K}{\partial t} = \frac{\partial K}{\partial T} \frac{\partial T}{\partial t}
\end{equation}
\paragraph{Surface temperature derivative}
Therefore the final derivative that we need is $\partial T/\partial t$, where $T$ here is taken as the surface temperature.  Unfortunately this isn't an easy term to get at, since temperature for us is a lookup quantity since entropy $S$ is the primitive (solution) variable.  Furthermore, application of the parameterised ultra-thin thermal boundary at the surface will further complicate obtaining the surface temperature derivative.  Actually, I think we can get at the temperature change (without accounting for the parameterised boundary layer) using the following:
\begin{equation}
dS = \left( \frac{\partial S}{\partial T} \right)_P dT + \left( \frac{\partial S}{\partial P} \right)_T dP
\end{equation}
\begin{equation}
dS = \left( \frac{c_P}{T} \right) dT
\end{equation}
Therefore:
\begin{equation}
\frac{\partial S}{\partial t} = \left( \frac{c_P}{T} \right) \frac{\partial T}{\partial t}
\end{equation}
And hence:
\begin{equation}
\frac{\partial T}{\partial t} = \frac{\partial S}{\partial t} \left( \frac{T}{c_P} \right)
\end{equation}
%%%%
%%%%
%%%%
\subsubsection{Global mass of liquid and solid}
The current version of SPIDER computes the hydrostatic pressure profile \emph{a priori} based on the Adams-Williamson equation of state.  This means that for simplicity we can compute the liquid mass by multiplying the mass of a particular radial shell $d\hat{m}_i$, with some function $g$, and then sum:% the melt fraction $\phi$, and then sum:
\begin{equation}
\hat{M}^{\rm l} = \sum_i^N g_i \ d\hat{m}_i, \qquad \hat{M}^{\rm s} = \sum_i^N (1-g_i) \ d\hat{m}_i
%\hat{M}^{\rm l} = \sum_i^N \phi_i \ d\hat{m}_i, \qquad \hat{M}^{\rm s} = \sum_i^N (1-\phi_i) \ d\hat{m}_i
\end{equation}
Note that $d\hat{m}_i$ is constant and computed from the Adam-Williamson equation of state.  \myemph{Formally, there is an inconsistency, since the lookup tables determine the density of the liquid and solid phases, but these will not necessarily give a constant mantle mass for all time!  This is why I choose to use the hydrostatic pressure profile instead, because by construction this will ensure the total mantle mass is unchanging during the course of a model run.}.  Also note that by construction Eq.~\ref{eq:mantle_mass} holds.
The derivative of liquid mass with respect to time is given by the chain rule:
\begin{equation}
\frac{d \hat{M}^{\rm l}}{dt} = \sum_i^N \frac{d g_i}{d S_i} \frac{dS_i}{dt} dm_i, \qquad \text{ for } 0 < \phi < 1
\end{equation}
The change in solid mass can be similarly computed, and since we are dealing with only two phases:
\begin{equation}
\frac{d \hat{M}^{\rm l}}{dt} = -\frac{d \hat{M}^{\rm s}}{dt}
\end{equation}
In the code, I experimented with smoothing across the liquidus and solidus for these integrated mass quantities, but early testing suggested that no smoothing performed better. A useful derivative for the temperature--pressure version of Spider is:
\begin{equation}
\frac{\partial \phi}{\partial r} = \frac{1}{\Delta T_{fus}} \left[\frac{\partial T}{\partial r} - \phi \frac{\partial T_{liq}}{\partial r} + (\phi-1) \frac{\partial T_{sol}}{\partial r} \right]
\end{equation}
%%%%
%%%%
%%%% below were experiments before utilising the fact that a constant mixing length prevents a lid from forming
%%%% also, preliminary tests did not seem to conserve volatile abundance (perhaps due to set accuracy of atmosts timestepper?)
%%%%
%%%%
%\subsubsection{Batch crystallisation}
%If we assume that the magma ocean batch crystallises, then:
%\begin{equation}
%g_i(t) = \phi_i(t)
%\end{equation}
%Basically, the ability of a partial of mass to retain volatiles is only dependent on its melt fraction.  Also:
%\begin{equation}
%\frac{d \hat{M}^{\rm l}}{dt} = \sum_i^N \frac{d \phi_i}{d S_i} \frac{dS_i}{dt} dm_i = \sum_i^N \frac{1}{\Delta S_{{\rm fus}i}} \frac{dS_i}{dt} dm_i, \qquad \text{ for } 0 < \phi < 1
%\end{equation}
%%%%
%%%%
%\subsubsection{Fractional crystallisation}
%We can retain the same basic formulation as for batch crystallisation, but now add a weighting factor that penalises the ability of low melt fractions to store volatiles:
%\begin{equation}
%g_i(t) = \frac{\phi(t)}{2} \left(1+ \tanh\left(\frac{\phi(t)-\phi_c}{\phi_w} \right)\right)
%\end{equation}
%where $\phi(t)$ is melt fraction, $\phi_c$ the critical melt fraction, and $\phi_w$ a smoothing width.  Compute the time derivative:
%\begin{equation}
%\frac{d g_i}{d S_i} =\frac{1}{\Delta S_{i,fus}} \frac{1}{2\phi_w} \frac{1}{\cosh^2{x_i}}, \qquad x_i = \frac{\phi_i(t)-\phi_c}{\phi_w}
%\end{equation}
%%%%
%%%%
%%%%
\subsection{Grey atmosphere model}
Given the mass of volatiles in the atmosphere, we can compute an effective emissivity of the atmosphere using a grey atmosphere model.  The model is described in detail in \cite{AM85} in the appendix, and the key equations are presented more clearly in \cite{ET08}.  Note also that \cite{AND10} describes the model in Sect. 3.7.2 in his book, but his formulation is slightly different from \cite{AM85} as he considers the boundary condition at the top of the atmosphere in terms of long-wave only.  \cite{AND10} considers the \myemph{net upward long wave irradiance}:
\begin{equation}
F_z = F_{\uparrow}^l - F_{\downarrow}^l = const, \qquad F_{\downarrow}^l=0, \qquad F_\uparrow^l(0)=F_0, \qquad \implies F_z=F_0
\end{equation}
Whereas \cite{AM85} consider the \myemph{net upward flux}, which includes both short and long wave components:
\begin{equation}
F_{atm} = F_{\uparrow} - F_{\downarrow} = const, \qquad F_{\downarrow}=F_\infty, \qquad \implies F_{\uparrow} = F_{atm} + F_\infty
\end{equation}
The incoming short wave irradiation is $F_\infty$ in \cite{AM85} and $F_0$ in \cite{AND10}.  And $F_{\uparrow}^s=0$ since reflection is not considered (by either author).

\subsubsection{Optical depth}
For a given volatile we compute the optical depth (and remember that optical depth is a non-dimensional quantity) \citep[Eq. A18,][]{AM85}:
\begin{equation}
\tau^\ast = \frac{3 \kappa^\prime p(\tau^\ast)}{2g}
%\tau^\ast = \frac{3 m_{\rm v}^{\rm g}}{8 \pi R_{\rm p} ^2} \sqrt{\frac{k_{\rm abs} g}{3 p_0}} = \frac{3p}{2g} \sqrt{\frac{k_{\rm abs} g}{3 p_0}} = \frac{3 p k_{\rm abs}^\prime}{2g}
\label{eq:tau}
\end{equation}
Remember that Henry's law gives us the \myemph{surface partial pressure}, so we can use this directly to compute the \myemph{surface optical depth}.  We don't need to go via the atmospheric mass of the volatile, but if we wanted we could also use:
\begin{equation}
\tau^\ast_s = \frac{3 \kappa^\prime m_{\rm v}^{\rm g}}{8 \pi R_p^2} \left( \frac{\mu_{\rm t}}{\mu_{\rm v}^{\rm g}} \right)
\end{equation}
Again, the molar mass ratio does not seem to appear in other author's formulations.  Non-dimensionalising:
\begin{equation}
\tau^\ast_s = \frac{ 3 \hat{m}_{\rm v}^{\rm g} \cdot 4 \pi M_0}{8 \pi \hat{R}_p^2 R_0^2} \frac{V_0}{10^6} \left( \frac{\hat{\mu}_{\rm v}^{\rm g}}{\hat{\mu}_{\rm t}} \right) \left[ \frac{\hat{k}_{\rm abs} R_0^2 \hat{g} S_0 T_0}{3 \hat{p}_0 P_0 M_0 R_0} \right]^{\frac{1}{2}}
\end{equation}
And since (reassuringly) all the dimensional scalings cancel:
\boxedeq{}{\tau^\ast_s = \frac{3}{2} \frac{V_0}{10^6} \frac{\hat{m}_{\rm v}^{\rm g}}{\hat{R}_{\rm p}^2} \left( \frac{\hat{\mu}_{\rm v}^{\rm g}}{\hat{\mu}_{\rm t}} \right) \sqrt{\frac{\hat{k}_{\rm abs} \hat{g}}{3 \hat{p}_0}}}
\subsubsection{Effective emissivity}
Finally, the optical depths for each volatile $\tau_j$  are combined to give an effective emissivity, which can then be fed into the usual formulation for a grey-body \citep[Eq. A14,][]{AM85}:
\begin{equation}
\epsilon = \frac{2}{\sum_j \tau_j^\ast +2}
\end{equation}
\subsubsection{Atmosphere structure (1-D)}
The temperature profile, as a function of optical depth, is \citep[Eq.~A15,][]{AM85}:
\begin{equation}
T(\tau^\ast) = \left( T_0^4 \frac{(\tau^\ast+1)}{2} +T_\infty^4 \right)^\frac{1}{4}
\label{eq:Ttau}
\end{equation}
where:
\begin{equation}
T_0 = \left( \frac{F_{atm}}{\sigma} \right)^\frac{1}{4}
\end{equation}
This only depends on the (scaled) optical depth, which is an effective optical depth obtained by addition of the optical depth of each volatile.

\subsubsection{Optical depth to height above planetary surface}
\begin{equation}
\rho = \frac{n \mu}{V}
\end{equation}
where $n$ is number of moles, $\mu$ molar mass, and $V$ volume. Ideal gas law:
\begin{equation}
pV = nRT \implies \rho = \frac{p \mu}{R_g T}
\end{equation}
Hydrostatic equilibrium:
\begin{equation}
\frac{dp}{dz} = - \rho g \implies \frac{dp}{p} = -\frac{\mu g}{R_g T}dz
\end{equation}
If you assume that $T$ is constant (or nearly constant), then the above is trivial to integrate.  But we also have a 1-D temperature of profile that we would like to honour, so we have to do more work.  Using Eq.~\ref{eq:tau}:
\begin{equation}
\frac{dp}{p} = \frac{d \tau^\ast}{\tau^\ast} \implies \frac{d \tau^\ast}{\tau^\ast} = -\frac{\mu g}{R_g T}dz
\end{equation}
We also know $T(\tau^\ast)$ from Eq.~\ref{eq:Ttau}, and therefore:
\begin{equation}
\frac{dz}{d\tau^\ast} = \frac{1}{\tau^\ast \beta} \left( T_0^4 \frac{(\tau^\ast+1)}{2} +T_\infty^4 \right)^\frac{1}{4}, \qquad \text{where } \beta = -\frac{\mu g}{R_g}
\label{eq:dzdtau}
\end{equation}
Everything on the RHS (except $\tau$) is known.  The \myemph{combined optical depth} at the surface is known from Eq.~\ref{eq:tau}, which gives an initial value:
\begin{equation}
\tau^\ast (z=0) = \tau^\ast_s
\label{eq:dzdtau2}
\end{equation}
We can take $\mu$ as the mean molecular weight of the atmosphere, i.e. molecular weights of H$_2$O and CO$_2$, weighted by the mixing ratio of each species in the atmosphere.  We can now solve for $z(\tau^\ast)$ using Eqs.~\ref{eq:dzdtau} and \ref{eq:dzdtau2}, and thus provide a mapping from the (aggregate scaled) optical depth to the vertical height coordinate above the planetary surface.  We (PS and I) attempted to solve the equations analytically using Mathematica, but we have thus far been unable to obtain a Real valued function only.  But testing with Mathematica and Python reveal that the equation is trivial to integrate numerically, so I instead implemented a RK4 algorithm (actually, Simpson's rule, since the ODE is only a function of $\tau$) in SPIDER.
\subsubsection{Flux and $T_{eqm}$}
The formulation in SPIDER involves specifying (or calculating) an effective temperature $T_{eqm}$, but this relates to the incoming stellar flux:
\begin{equation}
F_{sun} = \sigma T_{eqm}^4 = (1-\alpha) \frac{F_0^\prime}{D^2}
\end{equation}
where $\alpha$ is the bolometric albedo (usually around 0.2), $F_0^\prime$ is the averaged solar constant over the surface, and $D$ the planet--star distance (AU).
\begin{equation}
F_0^\prime = \frac{F_0}{4}
\end{equation}
where $F_0$ is the solar constant:
\begin{equation}
F_{0,t} = F_{0,t=0}^\prime \left[ 1 + 0.4 \left( 1 - \frac{t}{t_0} \right) \right] ^ {-1}
\end{equation}