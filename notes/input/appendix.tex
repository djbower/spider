\section{Useful mathematics}
\subsection{Finite Differences}
Second order accurate central finite differences for an even regular mesh give the following relationships.\\ \\ First derivative:
\begin{equation}
\frac{dS}{dr} \approx \frac{dS_{\rm{i}}}{dr} = \frac{S_{\rm{i+1}}-S_{\rm{i-1}}}{2 \Delta r}
\end{equation}
Second derivative:
\begin{equation}
\label{eqn:dsq}
\frac{d^2S}{dr^2} \approx \frac{d^2S_{\rm{i}}}{dr^2} = \frac{S_{\rm{i+1}}-2S_{\rm{i}}+S_{\rm{i-1}}}{\Delta r^2}
\end{equation}
Derivative of a spatially-dependent coefficient multiplying a gradient:
\begin{equation}
\label{eqn:selfadjoint}
\frac{d}{dr} \left( \alpha \frac{dS}{dr}\right) \approx \frac{d}{dr} \left( \alpha_i \frac{dS_i}{dr} \right) = \frac{1}{2 \Delta r^2} \left[ (\alpha_{\rm{i-1}}+\alpha_{\rm{i}})(S_{\rm{i-1}}-S_{\rm{i}}) + (\alpha_{\rm{i+1}}+\alpha_{\rm{i}})(S_{\rm{i+1}}-S_{\rm{i}})  \right]
\end{equation}
Eq.~\ref{eqn:selfadjoint} reduces to Eq.~\ref{eqn:dsq} for constant $\alpha$.
%%%%
%%%%
\subsection{Define equilibrium conditions}
\textbf{Devised by Patrick Sanan to address the problem of ensuring chemical equilibrium for volatiles that react in the magma ocean}

Assume that we have some time-dependent equilibrium law that must be obeyed, between quantities $A$ and $B$ representing species of matter in a system.
\begin{equation}\label{eq:equilibrium}
f(A,B,t) = 0
\end{equation}
Further assume that we have some way of converting $A$ to $B$ in the domain to maintain this equilibrium. We may exchange one unit of $A$ for $\beta$ units of $B$.  Finally, suppose that input rates of the two species to the system are given, as $\dot A_\text{in}$ and $\dot B_\text{in}$.  Since equilibrium must be instantaneously maintained, there is a rate at which the input material is converted from $A$ to $B$. We call this rate $\dot c$.  Thus, the rates of change of the quantities in the reservoir are
\begin{align}\label{eq:inputconvert}
  \dot A_\text{in} - \dot c = \frac{\partial A}{\partial t} \\
  \dot B_\text{in} + \beta \dot c = \frac{\partial B}{\partial t} \\
\end{align}
$\dot c$ may be eliminated, to give
\begin{equation} \label{eq:inputconvertelim}
  \dot A_\text{in} - \frac{\partial A}{\partial t} = \frac{1}{\beta}\left( \dot B_\text{in} -\frac{\partial B}{\partial t} \right)
\end{equation}
Equation \ref{eq:equilibrium} can be differentiated \footnote{$\partial_1f$ means the partial derivative of $f$ with respect to its first argument} to give
$$
\partial_1f(A,B,t) \frac{\partial A}{\partial t} + \partial_2f(A,B,t)\frac{\partial B}{\partial t} + \partial_3f(A,B,t) = 0
$$
and \ref{eq:inputconvertelim} may be used to eliminate, say  $\frac{\partial A}{\partial t}$, leaving a single ODE
$$
\partial_1f(A,B,t) \left[\dot A_\text{in} - \frac{1}{\beta}\left( \dot B_\text{in} -\frac{\partial B}{\partial t} \right) \right] + \partial_2f(A,B,t)\frac{\partial B}{\partial t} + \partial_3f(A,B,t) = 0
$$
(Note that $A$ still appears, but it could be expressed in terms of $B$ and $t$ via \ref{eq:equilibrium}).
Notes:
\begin{itemize}
  \item Could probably extend to multiple species.
  \item I don't see any reason that $f$ can't depend on other quantities.
  \item The main idea of interest here is to explicitly consider the instantaneous rate at which species are converted as they move in and out of a system in constant equilibrium. With $\beta =1$, hopefully this could also apply when we're talking about partitioning between the different reservoirs and removing material via various escape mechanisms.
\end{itemize}
