\subsection{Surface}
\subsubsection{Isothermal (constant entropy)}
To apply this boundary condition in terms of fluxes, the heat leaving the uppermost cell must balance the heat entering the cell plus any internal heat generation:
\begin{equation}
\frac{dQ}{dt} = E_{1} - E_{top} + V \rho H = 0
\end{equation}
Rearrange to find $E_{top}$:
\begin{equation}
E_{top} = E_1 + V \rho H
\end{equation}
Equivalently, in terms of fluxes:
\begin{equation}
J_{top} = \frac{1}{A_{top}} \left( J_1A_1 + V \rho H \right)
\end{equation}
%%%
\subsubsection{Ultra-thin thermal boundary layer parameterisation}
The thermal boundary layer at the surface of a magma ocean is expected to be very thin (a few cms) with a temperature drop of several hundred kelvin.  A mesh with 1 mm spacing is typically required to resolve the boundary layer which is only practical for a 1-D numerical model such as the one we are presenting.  However, it is computationally inefficient and unnecessary to retain such high resolution in the boundary layer when the magma ocean is solidifying and growing a much thicker thermal boundary layer (eventual thickness around 100 km) due to solid-state convection.  Therefore, even for 1-D models there is an advantage in parameterising the temperature drop across the ultra-thin thermal boundary layer such that a coarse mesh can be used.  Parameterising the boundary layer is the only practical option for 2-D dynamic models \citep[e.g.,][]{LT16}.  From mixing length theory, heat flux is:
\begin{equation}
J_q = \rho c_p \kappa_{\rm h} \left[ \frac{\partial T}{\partial z}-\left( \frac{\partial T}{\partial z} \right)_s \right] + \rho c_p \kappa \frac{\partial T}{\partial z}
\label{eq:Jq}
\end{equation}
For inviscid case \citep{ABE93}:
\begin{equation}
\kappa_{\rm h} = \sqrt{\frac{\alpha g l^4}{16} \left[ \frac{\partial T}{\partial z}-\left( \frac{\partial T}{\partial z} \right)_s \right] }
\label{eq:kappah_invisc}
\end{equation}
For viscous case \citep{ABE93}:
\begin{equation}
\kappa_h = \frac{\alpha g l^4}{18 \nu} \left[ \frac{\partial T}{\partial z}-\left( \frac{\partial T}{\partial z} \right)_s \right]
\label{eq:kappah_viscous}
\end{equation}
Recall that:
\begin{equation}
\left( \frac{\partial T}{\partial z} \right)_s = \frac{\alpha T}{\rho c_p} \left( \frac{\partial P}{\partial z} \right)
\end{equation}
Can rearrange Eqn.~\ref{eq:Jq} to solve for $\frac{\partial T}{\partial z}$using a root-finding algorithm (e.g., fsolve in scipy, python) given $J_q$ and depth (which is equal to the mixing length, $l$).
\begin{equation}
\label{eq:nonlinear}
\frac{J_q}{\rho c_p} - \kappa \frac{\partial T}{\partial z} - \kappa_h \left[ \frac{\partial T}{\partial z} - \left( \frac{\partial T}{\partial z} \right)_s \right] = 0
\end{equation}
With the ability to determine $\frac{\partial T}{\partial z}$, we can compute the temperature profile (as a function of depth) by integrating from a surface temperature  downwards (assume all other parameters, e.g., $\alpha$, $g$, etc. are constant for convenience, although properties can be computed at the current depth).  In either case the pressure range is negligible so the quantities are calculated at 0 GPa.  Since eddy diffusivity depends strongly on the mixing length, near the surface it is derived using the viscous scaling (Eqn.~\ref{eq:kappah_viscous}) even though the flow in the magma ocean away from the boundaries is inviscid.  Surface heat flux is:
\begin{equation}
J_q = F_{\rm top} = \sigma (T_s^4 - T_{\rm eqm}^4)
\end{equation}
We use the following parameters: $c_p=1000$ J/kg K, $\alpha=3 \times 10^{-5}$ K$^{-1}$, $\rho=3000$ kg/m$^3$, $g=10$ m/s$^2$, $\kappa=10^{-6}$ m$^2$/s, and $T_{\rm eqm}$ = 255 K.  Now we can compute the temperature drop across the upper thermal boundary layer ($\Delta T$) for a range of surface temperatures ($T_s$) (equivalently, heat flux $F_{\rm top}$).  Plotting $\Delta T$ versus $F_{\rm top}$ in log-log space results in approximately a straight line:
\begin{equation}
\log(\Delta T) = A \log( F_{\rm top} ) + B
\end{equation}
with best fit parameters $A=0.75834686$, $B=-1.71695869$.  Note that the intercept ($B$) is essentially the origin which physically is reasonable.  Also note that $A \approx 3/4$ \dbnote{Is this scaling the same/similar to boundary layer theory?}.  Therefore:
\begin{equation}
q_s \propto T_s^4 \propto (\Delta T)^{4/3}
\end{equation}
rearrange:
\begin{equation}
\Delta T = c T_s^3
\end{equation}
which gives an estimate of the temperature drop across the thin ($\sim$cm) boundary layer using mixing length theory as a function of the surface temperature.  This scaling agrees with \cite{RS06} \dbnote{but they derive from boundary layer theory?}.  We can fit a curve to the model to determine that the constant of proportionality is $c=8.05154344{\rm E-08}$.  \dbnote{Paul Tackley calculates $c=1\rm{e}-7$ which is comparable.}
%%%%
\subsubsection{Formation of a viscous lid with conventional mixing length}
In 1-D we can technically resolve the ultra-thin thermal boundary layer, but this has its own challenges at least for cases that use the conventional definition of the mixing length as distance from the nearest boundary.  Test cases show that near the surface the temperature gradient is much larger than the slope of the liquidus.  Therefore, the surface always cools through the mixed phase region at earlier time than the region below.  This forms a high viscosity lid at the surface with thickness comparable to the smallest mesh spacing ($\sim$ 1 mm).  The lid insulates the magma ocean because the eddy diffusivity in the high viscosity lid is orders of magnitude smaller than the eddy diffusivity in the melt-dominated region below it.  This behaviour is not physically reasonable since it is expected that the surface of a magma ocean remains largely molten during crystallisation except when a buoyant crust forms.  In addition, lid formation greatly increases the number of internal time steps taken by the numerical solver and thus the time integration becomes prohibitively expensive.

Choosing to parameterise the ultra-thin thermal boundary (following the theory above) and thus forgo a high resolution mesh near the surface unfortunately does not result in more physically reasonable evolution.  The cooling trajectory from model initialisation until about 3.7 kyr (for a black body boundary condition) is acceptable but after this time the energy flow near the surface (at the second basic node) decreases because energy transport in the mixed-phase region is less efficient than in the melt region.  The reduction in flux limits energy transfer to the surface and because the surface cools efficiently by radiation the temperature near the surface (first staggered node) drops precipitously towards the equilibrium temperature of the planet.  Furthermore, the reduction in flux (at the second basic node) causes energy to build up beneath the bottleneck which reheats the region immediately below.  Therefore in both cases (resolved or parameterised ultra-thin thermal boundary layer), a lid forms at the surface that prevents the large energy flow in the melt-dominated region below from reaching the surface to radiate.

The (slightly ad-hoc) solution so far is to use a constant mixing length such as in case ``BUM'' in \cite{BSW18}.  Relative to the convention mixing length definition, a constant mixing length allows more transport near the surface which prevents (or at least mitigates) the formation of a very thin and highly viscous lid.  \dbnote{Also working on a foundering boundary layer scaling.}

\subsubsection{Foundering Upper Boundary Layer}
\dbnote{Still need to discuss with Aaron about this}
Magma ocean cooling proceeds all the way through the mixed phase region up to the rheological transition in the upper boundary layer.  At this point, the heat flux drops precipitously, as the stiff surface region forces the thermal diffusivity way down, throwing an insulating blanket over the magma ocean.  This introduces unphysical cycles into the evolution, where the total heat content of the mantle oscillates up and down as the UBL locks up, releases, and locks up again.  This behavior reveals an important piece of missing physics in standard mixing length theory, which fails to account for all the effects of phase transitions, and the role they play in a foundering solid crust.

Standard mixing length theory adequately captures the continuous foundering of a thin skin of cooler material at the upper boundary.
This is easily demonstrated by calculating the net flux predicted by mixing length theory and comparing it with the time averaged flux expected from a periodically foundering skin of cool material.  These two calculations produce identical flux predictions (to within near-unity constants of proportionality), indicating the physical reasonableness of the mixing length formulation.  It is crucial to note, however, that mixing length theory assumes that the cool skin and the underlying mantle are both in the same phase, so that there is no difference or contrast between the foundering skin and the bulk mantle material other than a temperature difference.  This does not hold true for a solidified crust, however, where the crust must be throughly heated to provide the necessary latent heat to remelt it.  Furthermore, there is a strong rheological contrast between the crust and the underlying mantle, which enables it to persist as a single coherent unit as it sinks down through the mantle after foundering.  This introduces new physics into the problem which must be incorporated into our mixing length model in order to accurately capture the behaviour near the rheological transition.
\begin{itemize}
\item Foundering Crust Thickness
\item Neutral Buoyancy Sinking Depth
\item Crust Remelting Depth
\item Lower Bound on Mixing Length
\end{itemize}
%%%
%%%
%%%
\subsection{Core-mantle boundary}
\subsubsection{Simple core cooling}
Energy balance for the core, where internal heat sources for the core are excluded:
\begin{equation}
\frac{dQ_c}{dt} = m_c c_c \frac{dT_{core}}{dt} = -A_{cmb} J_{cmb} = -E_{cmb}
\label{eq:core_energy_balance}
\end{equation}
Energy balance for the lowermost element of the mesh that neighbours the core:
\begin{equation}
\frac{dQ}{dt}= V \rho c \frac{dT_{cmb}}{dt} = E_{cmb} - E_1 + V \rho H 
\end{equation}
Let:
\begin{equation}
T_{core} = \hat{T}_{core} T_{cmb}
\end{equation}
Therefore:
\begin{equation}
\frac{dT_{core}}{dt} = \hat{T}_{core} \frac{dT_{cmb}}{dt}
\end{equation}
Rearrange and use Eq.~\ref{eq:core_energy_balance} to eliminate $dT_{core}/dt$:
\begin{equation}
\frac{dT_{cmb}}{dt} = \frac{1}{\hat{T}_{core}} \frac{dT_{core}}{dt} = -\frac{E_{cmb}}{\hat{T}_{core} m_c c_c}
\end{equation}
Now return to energy balance in lowermost element, and eliminate $dT_{cmb}/dt$ using above:
\begin{equation}
- \frac{V \rho c E_{cmb}}{\hat{T}_{core} m_c c_c} = E_{cmb} - E_1 + V \rho H 
\end{equation}
Finally, rearrange to express $E_{cmb}$ in terms of $E_1$:
\begin{equation}
E_{cmb} = \frac{E_1 - V \rho H}{\left(1+\frac{V \rho c}{\hat{T}_{core} m_c c_c}\right)} = (E_1 - V \rho H)\left(1+\frac{V \rho c}{\hat{T}_{core} m_c c_c}\right)^{-1}
\end{equation}
Equivalently, in terms of fluxes:
\begin{equation}
J_{cmb} = \left( \frac{1}{A_{cmb}} \right) \frac{J_1A_1 - V \rho H}{\left(1+\frac{V \rho c}{\hat{T}_{core} m_c c_c}\right)} = \left( \frac{1}{A_{cmb}} \right) (J_1A_1 - V \rho H)\left(1+\frac{V \rho c}{\hat{T}_{core} m_c c_c}\right)^{-1}
\end{equation}
%%%
\subsubsection{Simple core cooling II}
An issue with the simple core cooling model is that we are considering the element that neighbours the core.  If the heat out of this element (say, due to gravitational separation) is large (because the basic node is in the mixed phase region), this can result in unrealistic cooling of the element.  Hence it is preferred to instead consider the actual boundary of the CMB (the lowermost basic node) when constructing the boundary condition.  As before:
\begin{equation}
\frac{dQ_c}{dt} = m_c c_c \frac{dT_{core}}{dt} = -A_{cmb} J_{cmb} = -E_{cmb}
\label{eq:core_energy_balanceII}
\end{equation}
But now express in terms of the entropy change at the CMB.  It's probably better to think of this as the entropy change in the mantle some small distance from the CMB on the mantle side:
\begin{equation}
\frac{dQ_c}{dt} = m_c c_c \hat{T}_{core} \frac{T_{cmb}}{c_{cmb}} \left( \frac{dS_{cmb}}{dt} \right) = -A_{cmb} J_{cmb} = -E_{cmb}
\end{equation}
Recall that $E_{cmb}$ is positive for radially outward (upward) flux.  Using our standard linear reconstruction we can relate the entropy gradient at the CMB to the entropy at the CMB:
\begin{equation}
S_{cmb} = S_{-1} + \frac{\Delta r}{2} \left( \left. \frac{dS}{dr} \right |_{cmb} \right)
\end{equation}
Since $\Delta r$ and $dS/dr$ are (generally) negative the second term is positive, and hence the entropy at the CMB is (generally) expected to be larger than the last staggered node value ($S_{-1}$).  Taking the time derivative and substituting in:
\begin{equation}
m_c c_c \hat{T}_{core} \frac{T_{cmb}}{c_{cmb}} \left( \frac{dS_{-1}}{dt} + \frac{\Delta r}{2} \frac{d}{dt} \left( \left. \frac{dS}{dr} \right |_{cmb} \right)  \right) = -A_{cmb} J_{cmb} = -E_{cmb}
\end{equation}
Rearranging:
\begin{equation}
\left( \frac{dS_{-1}}{dt} + \frac{\Delta r}{2} \frac{d}{dt} \left( \left. \frac{dS}{dr} \right |_{cmb} \right)  \right) = -E_{cmb} \left( \frac{c_{cmb}}{c_c} \right) \left( \frac{1}{m_c \hat{T}_{core} T_{cmb}} \right)
\end{equation}
\begin{equation}
\frac{\Delta r}{2} \frac{d}{dt} \left( \left. \frac{dS}{dr} \right |_{cmb} \right) = -E_{cmb} \left( \frac{c_{cmb}}{c_c} \right) \left( \frac{1}{m_c \hat{T}_{core} T_{cmb}} \right) - \frac{dS_{-1}}{dt}
\end{equation}
Now, $dS_{-1}/dt$ is the change in entropy of the last staggered node, which is given by the ``master'' equation since it depends on fluxes in and out of the volume as well as interior heat sources.
\begin{equation}
\frac{\Delta r}{2} \frac{d}{dt} \left( \left. \frac{dS}{dr} \right |_{cmb} \right) = -E_{cmb} \left( \frac{c_{cmb}}{c_c} \right) \left( \frac{1}{m_c \hat{T}_{core} T_{cmb}} \right) - \left( \frac{-Eout_{-1}+Ein_{-1}+\rho V H_{-1}}{\rho T V} \right)
\end{equation}
Now, $Ein_{-1}=E_{cmb}$:
\begin{equation}
\frac{\Delta r}{2} \frac{d}{dt} \left( \left. \frac{dS}{dr} \right |_{cmb} \right) = -E_{cmb} \left( \frac{c_{cmb}}{c_c} \right) \left( \frac{1}{m_c \hat{T}_{core} T_{cmb}} \right) - \frac{E_{cmb}}{\rho T V} - \left( \frac{-Eout_{-1}+\rho V H_{-1}}{\rho T V} \right)
\end{equation}
Also $Eout_{-1}$ and $H_{-1}$ are both known since they lie exclusively in the magma ocean domain.  In the magma ocean, $E_{cmb}$ can be computed for known $S_{cmb}$ and $dS/dr|_{cmb}$ (given by initial condition to start with).  Crucially, this will enable us to cut off gravitational separation as soon as the CMB drops below the solidus.
%%%
\subsubsection{Isothermal (constant entropy)}
To apply this boundary condition in terms of fluxes, the heat leaving the lowermost cell must balance the heat entering the cell plus any internal heat generation:
\begin{equation}
\frac{dQ}{dt} = E_{cmb} - E_1 + V \rho H = 0
\end{equation}
Rearrange to find $E_{cmb}$:
\begin{equation}
E_{cmb} = E_1 - V \rho H
\end{equation}
Equivalently, in terms of fluxes:
\begin{equation}
J_{cmb} = \frac{1}{A_{cmb}} \left( J_1A_1 - V \rho H \right)
\end{equation}