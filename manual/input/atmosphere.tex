\fbox{\parbox{\textwidth}{Dimensional and non-dimensional equation for the mass balance and evolution of mass balance of a volatile.}}

\noindent \dbnote{TODO: throughout this section the gravity is assumed to be positive, but in SPIDER it is actually negative!  So be aware of sign differences between these notes and the code.}

\subsection{Volatile Mass Balance}
The mass balance of a given volatile in the interior of a SPIDER model \citep[e.g.,][]{LMC13} is:
\begin{equation}
m_{\rm v}^{\rm s} + m_{\rm v}^{\rm l} + m_{\rm v}^{\rm g} + m_{\rm v}^{\rm e} + m_{\rm v}^{\rm o} + m_{\rm v}^r= m_{\rm v}^{\rm t}
\end{equation}
where subscript $v$ denotes a particular volatile and superscripts $s$, $l$, $g$, $o$, $t$ indicates the volatile's mass in the solid, liquid (i.e., melt), gas, and surface liquid ocean, as well as the total mass, respectively.  Superscript $e$ represents the reservoir (lost) due to escape, superscript $o$ accounts for ocean formation (this is just a placeholder and is not currently implemented), and superscript $r$ accounts for the amount of the volatile added (or removed) by chemical reactions.  We can now express these masses as follows:
\begin{equation}
X_{\rm v}^{\rm s} M^{\rm s} + X_{\rm v}^{\rm l} M^{\rm l} + X_{\rm v}^{\rm g} M^{\rm g} + m_{\rm v}^{\rm e} + m_{\rm v}^{\rm o} + m_{\rm v}^r = X_{\rm v}^{\rm init} M^{\rm m}
\end{equation}
where $X$ are mass fractions of volatile in each phase (solid, liquid, and gas), relative to the respective \myemph{physical} reservoir size of solid, liquid, and gas.  \myemph{In SPIDER we actually work with scaled volume or mass, i.e. the $4 \pi$ prefactor associated with spherical geometry is omitted for all spherical volume and mass quantities (except for output, where the $4 \pi$ is reintroduced).  Nevertheless, the above equation is valid for physical or scaled masses.}. The total mass of the mantle:
\begin{equation}
M^{\rm m} = M^{\rm s} + M^{\rm l}
\end{equation}
This is used because it is most sensible to define an initial mass fraction of volatiles relative to the total mantle mass, which is constant, even though the masses of the solid and liquid mantle evolve with time as the magma ocean cools and crystallises.  We assume a simple partition coefficient that relates the mass fraction of the volatile in the solid phase to the mass fraction of the volatile in the liquid phase.
\begin{equation}
k_{\rm v} = \frac{X_{\rm v}^{\rm s}}{X_{\rm v}^{\rm l}}
\end{equation}
We should now consider the \myemph{physical} atmospheric mass of a particular volatile.  First, consider the total atmospheric mass as being composed of $q$ species:
\begin{equation}
m_{\rm t}^{\rm g} = m_0^{\rm g} + m_1^{\rm g} + m_2^{\rm g} + \dots + m_q^{\rm g} = \frac{4 \pi R_p^2}{g} P_s
\end{equation}
where $R_{\rm p}$ is the planetary radius, and $P_s$ is the surface pressure.  Now express in terms of molar mass $\mu$:
\begin{equation}
\mu_{\rm t}^{\rm g} N = \mu_0^{\rm g} n_0 + \mu_1^{\rm g} n_1 + \mu_2^{\rm g} n_2+ \dots + \mu_q^{\rm g} n_q = \frac{4 \pi R_p^2}{g} P_s
\end{equation}
where $\mu_{\rm t}^{\rm g}$ is the mean molar mass of the atmosphere, $N$ the total number of moles, and $n_q$ is the number of moles of species $q$.  Now divide through:
\begin{equation}
\frac{\mu_0^{\rm g}}{\mu_{\rm t}^{\rm g}} \frac{n_0}{N} + \frac{\mu_1^{\rm g}}{\mu_{\rm t}^{\rm g}} \frac{n_1}{N} + \frac{\mu_2^{\rm g}}{\mu_{\rm t}^{\rm g}} \frac{n_2}{N}+ \dots + \frac{\mu_q^{\rm g}}{\mu_{\rm t}^{\rm g}} \frac{n_q}{N} = 1
\end{equation}
The definition of partial pressure $p_q$:
\begin{equation}
\frac{n_q}{N} = \frac{p_q}{P_s}
\end{equation}
Note that the partial pressure must vary with optical depth (i.e., height), but for a well-mixed atmosphere the ratio of the partial pressure to the total pressure for a given species is constant.  By reintroducing the constant factors leads to:
\begin{equation}
4 \pi R_p^2 \left( \frac{\mu_0^{\rm g}}{\mu_{\rm t}^{\rm g}} \right) \frac{p_0}{g} + 4 \pi R_p^2 \left( \frac{\mu_1^{\rm g}}{\mu_{\rm t}^{\rm g}} \right) \frac{p_1}{g} + \dots + 4 \pi R_p^2 \left( \frac{\mu_q^{\rm g}}{\mu_{\rm t}^{\rm g}} \right) \frac{p_q}{g} = \frac{4 \pi R_p^2 P_s}{g} = m_{\rm t}^{\rm g}
\end{equation}
Demonstrating that the mass of a given volatile species $q$ is related to the \myemph{surface partial pressure} as:
\begin{equation}
m_q^{\rm g} = 4 \pi R_p^2 \left( \frac{\mu_q^{\rm g}}{\mu_{\rm t}^{\rm g}} \right) \frac{p_q}{g}
\end{equation}
Many previous studies that consider outgassing of multiple volatiles species do not use this correct expression \citep[e.g.,][]{ET08,LMC13,SMD17,NKT19}.  This is discussed in \cite{BKW19}.  The \myemph{physical} atmospheric mass of a particular volatile is given by:
\begin{equation}
m_{\rm v}^{\rm g} = X_{\rm v}^{\rm g} M^{\rm g} = \frac{4 \pi R_{\rm p}^2}{g} \left( \frac{\mu_{\rm v}^{\rm g}}{\mu_{\rm t}} \right) p (X_{\rm v}^{\rm l})
\end{equation}
\myemph{We must exclude the factor of $4 \pi$ for scaled mass!}.  $p(X_{\rm v}^{\rm l})$ is the partial pressure of the volatile which is a function of the mass fraction in the liquid phase, i.e., by a modified (power-law form) of Henry's law:
\begin{equation}
p( X_{\rm v}^{\rm l} ) = \left( \frac{X_{\rm v}^{\rm l}}{\alpha} \right)^\beta, \qquad \frac{dp}{d X_{\rm v}^{\rm l}} = \frac{\beta}{\alpha} \left( \frac{X_{\rm v}^{\rm l}}{\alpha} \right)^{\beta-1}
\label{eq:Henry_mod}
\end{equation}
where $\alpha$ and $\beta$ are parameters for each volatile.  The ``standard'' Henry's law is recovered when $\beta=1$, but allowing a power-law form provides more flexibility for volatiles that do not follow Henry's law exactly ($\beta \neq 1$).  \myemph{From now on we will only consider the scaled mass which omits the $4 \pi$ factor associated with spherical geometry:}
\boxedeq{eq:dimvolatile}{X_{\rm v}^{\rm l} (k_{\rm v} M^{\rm s} + M^{\rm l}) + \frac{R_{\rm p}^2}{g} \left( \frac{\mu_{\rm v}^{\rm g}}{\mu_{\rm t}} \right) p (X_{\rm v}^{\rm l}) + m_{\rm v}^{\rm e} + m_{\rm v}^{\rm o} + m_{\rm v}^r = X_{\rm v}^{\rm init} M^{\rm m}}
And note, importantly, that we solve for the volatile mass fraction in the liquid phase, from which we can subsequently compute the volatile mass in the solid and gas phase.  The above equation also shows that the volatile abundances are coupled through the mean molecular weight:
\begin{equation}
\mu_{\rm t} = \frac{1}{N} \sum_q \mu_q n_q = \frac{1}{P_s} \sum_q \mu_q p_q \qquad P_s = \sum_q p_q
\label{eq:atmosphere_molar_mass}
\end{equation}
%%%%
%%%%
\subsection{Chemical Reactions}
We use a chemical model similar to \cite{GS14}, wherein we assume chemical reactions are at equilibrium at each time step and then we calculate the change in concentration of reactants and products based on how much of any volatile is added to the system.  \textbf{Here, we are considering reactions between volatiles (i.e., gas phases) that are dissolved in a liquid phase (i.e., the melt, the magma ocean with 100\% melt fraction)}.
\subsubsection{Water production in the magma ocean}
For the reaction that produces water:
\begin{equation}
    {O_2} + 2H_2 \leftrightarrow 2H_2O
    \label{eq:altreaction} 
\end{equation}
The equilibrium constant for this reaction, \textbf{expressed in terms of concentrations (i.e., square brackets)} is:
\begin{equation}
    K = \frac{[H_2O]^2}{[H_2]^2[O_2]}
    %K_{eq} = \frac{(P_{H_2O})^2}{\left(P_{H_2}\right)^2 P_{O_2}}
\end{equation}
Note that the equilibrium constant should be unitless, although often this is not the case in the geophysical and astrophysical literature.
This is rewritten in terms of the oxygen fugacity $fO_2$, as: 
\begin{equation}
    \frac{[H_2O]}{[H_2]} = \sqrt{K_{eq}} \left(fO_2\right)^{1/2}
    %\frac{P_{H_2O}}{P_{H_2}} = \sqrt{K_{eq}} \left(fO_2\right)^{1/2}
\end{equation}
Note that the stoichiometry is important for defining the equilibrium constant.  If instead we consider the reaction:
\begin{equation}
    \frac{1}{2} O_2 + H_2 \leftrightarrow H_2O
    \label{eq:reaction}
\end{equation}
We instead derive:
\begin{equation}
     K_{eq} = \frac{[H_2O]}{[H_2] [O_2]^{1/2}} = \frac{[H_2O]}{[H_2] \left(fO_2\right)^{1/2}}
    \label{eq:Keq}
\end{equation}
%and therefore by rearranging:
%\begin{equation}
%    \frac{P_{H_2O}}{P_{H_2}} = K_{eq} \left(fO_2\right)^{1/2}
%\end{equation}
The equilibrium constant for this reaction is calculated based on the surface temperature using data from \cite{RBF78} and a fit from \cite{OS19}.  Note that there seems to be an inconsistency between the assumed stoichiometry and the equilibrium constant within the main body of the text in \cite{OS19}, but this appears to be clarified in their Table~3.  \textbf{Therefore, let's continue using $K_{eq}$ (Eq.~\ref{eq:Keq_OS19}), $fO_2$ (Eq.~\ref{eq:fO2_OS19}) and Eq.~\ref{eq:Keq}}.
%\tknote{I checked this because I also noticed the inconsistency with the reaction equation and expression of the equilibrium constant. Turns out that for the surface temperature range necessary (between 500 and 4000K is what I checked because it is the bounds of the model runs that I did before), both expressions give values that are very close. Converting their expressions for $K_{eq}$ (Eq. 246 here) and $fO_2$ (Eq 40 in Olson \& Sharp) into exponential form and putting them back into the chemical equilibrium expression gives: 
%$K_{eq}(fO_2) ^{1/2} = 10^{\textbf{7.39}\times10^5T_s^{-1.61} - 1.39\times10^6T_s^{-1.7}}$  \\
%or \\
%$(K_{eq} fO_2) ^{1/2} = 10^{\textbf{3.695}\times10^5T_s^{-1.61} - 1.39\times10^6T_s^{-1.7}}$\\
%The difference between the two is only a factor of 1/2 in an exponent that is of order $10^5$, so it doesn't change anything significantly--I can send you the plot but there also isn't much to see (because the lines overlap entirely). We can use either Eq 241 or 244.}
The equilibrium constant depends on the surface temperature \citep[Eq.~41,][]{OS19}:
\begin{equation}
    \log_{10} K_{eq} = 7.39\times 10^5T_s^{-1.61}
    \label{eq:Keq_OS19}
\end{equation}
The oxygen fugacity depends on the surface temperature \citep[Eq.~40,][]{OS19}:
\begin{equation}
\log_{10} fO_2 = -2.75 \times 10^6 T_s^{-1.7}
\label{eq:fO2_OS19}
\end{equation}
Combining Eq.~\ref{eq:Keq_OS19} and \ref{eq:fO2_OS19}:
\begin{equation}
K_{eq} (fO_2)^{1/2} = 10^{7.39 \times 10^5 T_s^{-1.61}-1.375 \times 10^6 T_s^{-1.7}}
\label{eq:Keq_fO2_OS19}
\end{equation}
Now we have two options, which we could in principle switch between in the code with a user-defined FLAG:
\begin{enumerate}
\item Compute the mean of Eq.~\ref{eq:Keq_fO2_OS19} over the surface temperature range from 500 to 4000 K to eliminate $T_s$:
\boxedeq{}{K_{eq}\left(fO_2\right)^{1/2} = 0.01 \label{eq:Keq_fO2_approx}}
\item Retain dependence on the surface temperature $T_s$, since this is computed (known) at every time step within the code:
\boxedeq{}{K_{eq} (fO_2)^{1/2} = 10^{7.39 \times 10^5 T_s^{-1.61}-1.375 \times 10^6 T_s^{-1.7}} \label{eq:Keq_fO2_Ts}}
\end{enumerate}
\textbf{For testing purposes, it's fine to stick to (1), but extension to (2) should not be difficult}.  There are other expressions that we could adopt in the future to determine $fO_2$.  The oxygen fugacity, which is determined by the iron-w\"{u}stite buffer, can be calculated using the interior pressure and temperature using either \citep{O87}: 
\begin{equation}
    \log_{10}\left(fO_2\right) = 6.899 - \frac{27714}{T} + \frac{0.05(P-1)}{T}
\end{equation}
or \citep{F91}:
\begin{equation}
    \log_{10}\left(fO_2\right) = 6.702-\frac{27489}{T} + \frac{0.055(P-1)}{T}
\end{equation}
However, for simplicity at the present time we use Eq.~\ref{eq:Keq_fO2_approx}.  Therefore, at equilibrium we have:
\boxedeq{}{\frac{[H_2O]}{[H_2]} = K_{eq} \left(fO_2\right)^{1/2}=0.01}
%\dbnote{My understanding now is that, given our simplifications, the concentration of H$_2$O to H$_2$ is fixed in the mantle.  Note that it would be trivial to reinstate the surface temperature $T_s$ as a controlling parameter, since we compute this every time step within the code.  So you don't need to average over the surface temperature range if you don't want to.}\tknote{Doing this now.}
%\dbnote{OK, so now we can easily compute $(fO_2)^{1/2}$ for any surface temperature, since it only depends on $K_{eq}$}\tknote{$fO_2$ is set by the pressure and temperature within the mantle and so should eventually be calculated using expression 247 or 248. $K_{eq}$ is presumably also determined by the interior temperature and pressure, because that is what changes the iron-wustite buffer, but for now 246 is the best expression I have found. }\\
%\dbnote{But this above expression is not explicitly used, since we instead compute $Q$ and compare to $K$, as described below?} \tknote{Eq. 250 is used to establish chemical equilibrium in the MO at every timestep. } \\
%\dbnote{Currently, this condition is not at all enforced in the volatile evolution.  I.e., outgassing is driven by over-saturation of the melt phase, and this results in changes in the partial pressure of species in the atmosphere.  The partial pressures are coupled through the fact that partial pressure of one species can change as a consequence of other species outgassing.  But there are no chemical reactions by default.} \\
%\dbnote{It's not obvious to me that we have to do all this work.  If we know the concentration in the magma ocean of, for example, He, then thermodynamic considerations of the reaction already constrain the concentration of H$_2$O.  So we might be able to solve directly, rather than iterating to determine $[\delta x]$.  This seems to be what Olson and Sharp (2019) do; they track He and then directly compute H$_2$O.}
There are other physical processes operating during the evolution of a magma ocean (e.g., outgassing, escape), such that we expect this reaction is driven away from equilibrium.  If the reaction is not in equilibrium, we use Eq.~\ref{eq:Keq} to calculate the reaction quotient Q:
\begin{equation}
Q\left(fO_2\right)^{1/2} = \frac{[H_2O]^\ast}{[H_2]^\ast }
\end{equation}
That is, \textbf{we use the current values of the concentrations of $H_2O$ and $H_2$ (denoted by $^\ast$) to compute the LHS, i.e., Q$(fO_2)^{1/2}$}.
%\dbnote{Now how do we know $(fO_2)^{1/2}$?  It is inconsistent to average over surface temperature as above and then use $K_{eq}$ with its $T_s$ dependence to back-compute $(fO_2)^{1/2}$. Now I think you are only averaging the $T_s$ for the $fO_2$ equation and retaining $K_{eq}$ with its full surface temperature dependence.  Please clarify.} \tknote{I averaged over $K_{eq}(fO_2)^{1/2}$ for the surface temperature range 500-4000K, so the RHS of 252. And then since $fO_2$ is fixed, I am actually comparing $K_{eq}(fO_2)^{1/2}$ to $Q(fO_2)^{1/2}$ because the comparisons still hold. }.
%, where $(fO_2)^{1/2}$ is known from Eq.~\ref{eq:fO2}  \dbnote{Basically, the partial pressures here are after outgassing, atmospheric escape etc., but before we do any chemistry}\tknote{Yes, they are the partial pressures within the mantle.}.
We can then compare Q$(fO_2)^{1/2}$ to K$_{eq}(fO_2)^{1/2}$ to determine if the system is in equilibrium and if it is not, how it will shift to go back into equilibrium. If the reaction quotient is not equal to the equilibrium constant, the concentrations of products will change by some amount $[\delta x]$:
\begin{equation}
K_{eq} (fO_2)^{1/2} = \frac{([H_2O]^\ast + [\delta x])}{([H_2]^\ast - [\delta x])}
\label{eq:Keq_diff}
\end{equation}
%\dbnote{Note that we don't need these three cases here.  $\delta x$ encapsulates sign information, to show whether we need more products or more reactants.  We just have to pick a convention.} \tknote{Set the forward reaction to be the default (i.e. converting hydrogen to water). I also had mixed up the comparison before, if $Q<K$ we produce products and if $Q>K$ we produce reactants.}. \dbnote{But this remains consistent with the text below, where $Q<K_{eq}$ means that there are more reactants and less products than what equilibrium wants---right?} \tknote{yes, I changed it already. I had written the opposite before, that Q<K means more products and less reactants.}
If $Q(fO_2)^{1/2} < K_{eq}(fO_2)^{1/2}$, there are more reactants and less products than what is required for equilibrium, and therefore the concentration of reactants decreases and products increase.  In this case, the $\delta x$ in Eq.~\ref{eq:Keq_diff} is positive. If $Q(fO_2)^{1/2} > K_{eq}(fO_2)^{1/2}$, reactants are produced and $\delta x$ is negative.  If $Q(fO_2)^{1/2} = K_{eq}(fO_2)^{1/2}$ the system is still in equilibrium.  At each time step, we calculate $Q(fO_2)^{1/2}$, compare to $K_{eq}(fO_2)^{1/2}$, and then adjust the concentrations of the volatiles to maintain equilibrium before continuing.  For the reaction above, it's actually trivial to compute $[\delta x]$ directly:
\begin{equation}
[\delta x] = \frac{K_{eq} (fO_2)^{1/2} \times [H_2]^\ast - [H_2O]^\ast }{K_{eq} (fO_2)^{1/2} + 1}
\end{equation}
%\dbnote{Agreed.  We want to solve for $\delta p$ and update the partial pressures accordingly.  Now because the partial pressures of each species relates to the concentration in the interior, the interior concentration is affected by these reactions.  I think to avoid time-reversing our reactions we have to do all the chemistry at the end of the time loop.  This might prevent us from including the chemistry as part of the same ODE that deals with the outgassing and escape.  I will have to think more about this.  Also, since we are dealing directly with partial pressures, maybe we can add a correction or extra term to an existing reservoir rather than add a new linear term in the mass balance.}
We can now compute the ``revised'' concentrations of [H$_2]^\dagger$ and [H$_2$O$]^\dagger$ according to:
\begin{equation}
[H_2O]^\dagger = [H_2O]^\ast + [\delta x], \qquad [H_2]^\dagger = [H_2]^\ast - [\delta x]
\end{equation}
If the concentrations are in terms of partial pressures (which I believe is the default), then we can use these revised concentrations (partial pressures) to determine the concentration of the volatiles in the magma ocean that are involved in the chemical reaction through the modified Henry's law:
\begin{equation}
X_{\rm v}^{\rm l} = \alpha p_v^{1/\beta} %, \qquad \frac{dp}{d X_{\rm v}^{\rm l}} = \frac{\beta}{\alpha} \left( \frac{X_{\rm v}^{\rm l}}{\alpha} \right)^{\beta-1}
\label{eq:Henry_mod_invert}
\end{equation}
where $p_v$ is the concentration of a given volatile in terms of its partial pressure (i.e., these are the $[]$ quantities above).  Hence $X_{\rm v}^{\rm l}$ is then the concentration (by mass) in the magma ocean.  \dbnote{OK.  Thing is, we technically want an update equation for the $X$'s, rather than re-setting their value explicitly.}\tknote{Would it make more sense to use $X_{\rm v}^{\rm l}$ for the equilibrium calculations instead of the partial pressure? At the end we would be left with $\delta X_{\rm v}^{\rm l}$? If we do this the value of the equilibrium constant might have to be altered, but I'm not sure by how much. Alternatively, if we stick with using the partial pressures and need just an update value, we could just take the difference before and after the reactions?: }
\begin{equation}  
    \delta X_{\rm v}^{\rm l}^\dagger = X_{\rm v}^{\rm l}^\ast - \alpha(p_v\pm\delta p)^{1/\beta} 
\end{equation}
\tknote{This way the updated amount of the mass fraction is simply the difference between the value before and after the chemistry. \\
Or another thing I tried is taking the time derivative of Eq. 258:}
\begin{equation}
    \frac{d  X_{\rm v}^{\rm l}}{dt} = \frac{\alpha}{\beta}p_{\rm v}^{1/\beta -1} \times \Big(\frac{dp_{\rmv}}{dt} \Big)^{1/\beta}
\end{equation}
\tknote{And then:} 
\begin{equation}
    \frac{dp_{\rmv}}{dt} = [\delta x]
\end{equation}
\tknote{is the updated partial pressure that we calculate and $p_{\rm v}$ is the partial pressure after the chemical reactions:} 
\begin{equation}
    p_{\rm v}=p_{\rm v}^\ast \pm [\delta x]
\end{equation}
\tknote{So the time evolution could be:} 
\begin{equation}
     \frac{d  X_{\rm v}^{\rm l}}{dt} = \frac{\alpha}{\beta}\left( p_{\rm v}^\ast \pm [\delta x] \right)^{1/\beta -1} [\delta x]^{1/\beta}
\end{equation}
\tknote{And then we would use this instead of} \ref{eq:chemevol}? 

As a technical detail, we probably need to compute the chemical reactions after other physical processes such as outgassing (if solidification of the magma ocean is being considered) and atmospheric escape.  Otherwise, there's a possibility we will simply remain locked at thermodynamic equilibrium and nothing interesting will happen (i.e., volatiles will not evolve).  \dbnote{Whether this is an issue or not will probably become more apparent later.} 

The volatile parameters \texttt{sign} and \texttt{coeff} are determined by the stoichiometry of the chemical reaction. The sign of the exponent indicates whether the volatile is a reactant (for which $\texttt{sign}=-1$) or a product ($\texttt{sign}=1$) in order to correctly compute Eq.~\ref{eq:Keq}. The coefficient is the constant of the chemical reaction. For example, in the reaction with water using equation \ref{eq:reaction} $\texttt{coeff}=1$ for hydrogen and water. However if we chose to use Eq.~\ref{eq:altreaction} instead, $\texttt{coeff}=2$ for hydrogen and water. 

The relationship between the partial pressure of a volatile added or subtracted due to chemical reactions and its mass evolution is: 
\boxedeq{}{\frac{dm_{\rm v}^r}{dt} = \pm C_{\rm v} \left( \frac{\delta p}{P_T} \right) \left( \frac{\mu_{\rm v}}{\mu_T} \right) M^l \label{eq:chemevol} }
where the sign is determined through the comparison of Q and K and whether the volatile is a product or reactant. $C$ is the coefficient from the chemical reaction and is also dimensionless.  

%%%%
%%%%
\subsubsection{Additional notes on \cite{OS19}}
An aspect I hadn't fully appreciated until now, is that in some sense H$_2$O is not a ``free'' volatile for \cite{OS19}.  Basically, the only way that H$_2$O can be produced is through chemical reactions involving H$_2$.  This means that the concentration of H$_2$O at the surface is solely dependent on the concentration of H$_2$ \citep[Eq.~39,][]{OS19}.  This addresses a source of confusion I was having, which is how you can ``independently'' evolve H$_2$ and H$_2$O according to their own solubility criteria when thermodynamic considerations of the interior demand that the concentrations are linked.  In this regard, I'm not sure we can honour the existing independent evolution of all volatiles in SPIDER whilst at the same time enforcing the thermodynamic condition.  Basically, there are too many (inconsistent) constraints.

Following \cite[Eq.~35,][]{OS19} (i.e., Dalton's law):
\begin{equation}
P_s = P_{H_2} + P_{He} + P_{H_2O}
\end{equation}

And the equilibrium condition \citep[Eq.~36,][]{OS19}:
\begin{equation}
\frac{P_{H_2O}}{P_{H_2}} = \epsilon
\end{equation}
Therefore:
\begin{equation}
P_s = P_{H_2} + P_{H_2O} + P_{He} = (1+\epsilon) P_{H_2} + P_{He} %= P_{H_2}^\ast + P_{He}
\end{equation}
And:
\begin{equation}
\frac{P_{H_2}}{P_{H_2} + P_{H_2O}} = \frac{1}{1+\epsilon}
\end{equation}
Therefore, relative to no chemical reaction occurring, the ratio of partial pressures drops below unity.
%where $P_{H_2}^\ast$ is the equivalent partial pressure of H$_2$ accounting for the reaction.

\dbnote{Continue with derivative for Eq.~37 in Olson and Sharp (2019).  I added some stuff below (now commented out in the latex version) but I don't think it's quite right --- the volatiles should always be independent of each other in terms of computing their partial pressure.}
where $P_{H_2,H_2O}$ is an effective partial pressure that accounts for both H$_2$ and H$_2$O (i.e., taking account of reactions).  This must relate directly to the total number of moles of H$_2$ in the atmosphere $x_{H_2}$ (again, including the contribution from H$_2$O):
\begin{equation}
P_{H_2}^\ast = x_{H_2} P_s
\end{equation}
Therefore:
\begin{equation}
P_{H_2} = \left( \frac{x_{H_2} P_s}{1+\epsilon} \right)
\end{equation}
Finally, $P_{H_2}$ is the partial pressure of $H_2$ in the absence of reactions, which is given by our modified Henry's law (Eq.~\ref{eq:Henry_mod}):
\begin{equation}
X_{H_2}^l = \alpha \left( \frac{x_{H_2} P_s}{1+\epsilon} \right) ^ {1/\beta}
\end{equation}
This completes the derivation of \cite[Eq.~37,][]{OS19} and \textbf{suggests that we can accommodate reactions by simply modifying (again) the form of Henry's law to account for an extra factor of $\epsilon$.}. \dbnote{but how to deal with $x_{H_2}$ since usually we don't consider this since it relates directly to the partial pressure of Henry's law (in the absence of chemical reactions).}

Now, importantly \cite{OS19} have a standard Henry's law for noble gases, but the concentration of H$_2$O is directly tied to H$_2$:
\begin{equation}
X_{H_2O}^l = 1.1 ( \epsilon X_{H_2}^l )^{1/2}
\end{equation}

You can consider this to be a type of Henry's law, in that the partial pressure of H$_2$O appears on the RHS, but the important point is that it is explicitly tied to the partial pressure of H$_2$ through the equilibrium reaction.

%%%%
%%%%
\subsubsection{Generalized Expressions}
More generally, a chemical reaction: 
\begin{equation}
    aA + bB \leftrightarrow cC + dD
\end{equation}
where $a,b,c,d$ are constants and $A,B,C,D$ are volatiles.  The equilibrium constant is: 
\begin{equation}
    K_{eq} = \frac{X_C^c X_D^d}{X_A^a X_B^b}
\end{equation}
and we calculate the difference in concentration using: 
\begin{equation}
K_{eq} = \frac{(X_C + C_C\delta x)^{C_C} (X_D + C_D\delta x)^{C_D}}{(X_A-C_A\delta x)^{C_A} (X_B-C_B\delta x)^{C_B}}
\end{equation}
Whichever value of $\delta x$ gives $Q=K$ determines the amount that the concentrations will change in the timestep. The value of $\delta x$ is usually a partial pressure, which must be converted to a mass fraction via Henry's law. 

\subsubsection{Newtons Method for finding $\delta x$}: 
Since we cannot generally solve for $\delta x$, as is the case in water production, we might need an iterative method to find the roots. Newtons method finds the roots of a function by adjusting a guess using the function and it's derivative. The expression for theequilibrium constant becomes: 
\begin{equation}
    f(\delta x) = \frac{(X_C + C_C\delta x)^{C_C} (X_D + C_D\delta x)^{C_D}}{(X_A-C_A\delta x)^{C_A} (X_B-C_B\delta x)^{C_B}} - K_{eq} 
\end{equation}
And it's derivative: 
\begin{equation}
\begin{aligned}
    f'(\delta x) =  & 
    \Big(\frac{(X_C + C_C\delta x)^{C_C} (X_D + C_D\delta x)^{C_D}}{(X_B-C_B\delta x)^{C_B}} \Big) -C_A^2    (X_A - C_A\delta x)^{-C_A-1} +  \\
    & \Big(\frac{(X_C + C_C\delta x)^{C_C} (X_D + C_D\delta x)^{C_D}}{(X_A-C_A\delta x)^{C_A}} \Big) -C_B^2   (X_B - C_B\delta x)^{-C_B-1} + \\
    & \Big( \frac{ (X_D + C_D\delta x)^{C_D}}{(X_A-C_A\delta x)^{C_A} (X_B-C_B\delta x)^{C_B}} \Big)  C_C^2   (X_C + C_C\delta x)^{ C_C-1} + \\
    & \Big( \frac{ (X_C + C_C\delta x)^{C_C}}{(X_A-C_A\delta x)^{C_A} (X_B-C_B\delta x)^{C_B}} \Big)  C_D^2   (X_D + C_D\delta x)^{ C_D-1}  
\end{aligned}
\end{equation}
Or, in a somewhat more condensed notation:
\begin{equation}
    f'(\delta x) = \mathlarger{\sum}_{i=1}^n {\rm sign}_iC_i^2(X_i+{\rm sign_iC_i\delta x})^{{\rm sign_i} C_i-1} \prod_{j\neq i}^n (X_j+{\rm sign}_i C_i \delta x)^{{\rm sign}_i C_i}
\end{equation}
Where ${\rm sign}_i$ is the same parameter as described before, and indicates whether the volatile is a product or reactant. The default convention is the forward reaction, so reactants have negative signs while products have positive signs. If the reaction were to occur backward, then $\delta x$ would simply be negative. The constant $C_i$ is the coefficient, described previously as well, and tells you how many molecules of a given species are involved in the reaction. $X_i$ can be taken to be partial pressures, or mass fractions within the mantle with the corresponding equilibrium constant. g

For Newton's method, we start with some guess $x_n$ of the root, and then compute a new approximation using: 
\begin{equation}
    x_{n+1} = x_n - \frac{f(x_n)}{f'(x_n)}
\end{equation}
and continue until our approximation of $x_n$ is consistent for some threshold (i.e. until $x_n-x_n+1<0.01$, or another similar condition). 

%%%%
%%%%
\subsection{Non-dimensionalisation}
\subsubsection{Mass}
In the code, we non-dimensionalise all masses as:
\begin{equation}
M = \rho_0 R_0^3 \hat{M} = M_0 \hat{M}
\end{equation}
Note that $\rho_0$ and $R_0$ are chosen to ensure that the solution quantities are around unity value.  They do not necessarily correspond to a physically meaningful value.  Therefore, $R_0$ is not necessarily the radius of the planet $R_p$.  $\hat{M}$ is the non-dimensional mass.  The non-dimensional mass balance is:
\begin{equation}
k_{\rm v} X_{\rm v}^{\rm l} \hat{M}^{\rm s} + X_{\rm v}^{\rm l} \hat{M}^{\rm l} + \frac{R_{\rm p}^2}{M_0 g} \left( \frac{\hat{\mu}_{\rm v}^{\rm g}}{\hat{\mu}_{\rm t}} \right) p (X_{\rm v}^{\rm l}) + \frac{m_{\rm v}^{\rm e}}{M_0} + \frac{m_{\rm v}^{\rm o}}{M_0} = X_{\rm v}^{\rm init} \hat{M}^{\rm m}
\end{equation}
Note that we have also non-dimensionalised the molar masses in this step.
\subsubsection{Volatile Concentration}
It's more convenient to express volatile concentration as a scaled version of parts-per-million (ppm).  This scaling is to ensure we can scale the volatile evolution equations similar to other quantities in the system of equations (like entropy, entropy gradient, etc.):
\begin{equation}
\hat{X} = \frac{X_{\rm ppm}}{V_0} = \frac{10^6 X_{\rm v}}{V_0}
\end{equation}
Therefore, the mass balance:
\begin{equation}
k_{\rm v} \frac{X_{\rm ppm}^{\rm l}}{V_0} \hat{M}^{\rm s} + \frac{X_{\rm ppm}^{\rm l}}{V_0} \hat{M}^{\rm l} + \frac{10^6}{V_0} \frac{R_{\rm p}^2}{M_0 g} \left( \frac{\hat{\mu}_{\rm v}^{\rm g}}{\hat{\mu}_{\rm t}} \right) p (X_{\rm v}^{\rm l}) + \frac{10^6}{V_0} \frac{m_{\rm v}^{\rm e}}{M_0} + \frac{10^6}{V_0} \frac{m_{\rm v}^{\rm o}}{M_0} = \frac{X_{\rm ppm}^{\rm init}}{V_0} \hat{M}^{\rm m}
\end{equation}
Collect terms:
\begin{equation}
\hat{X}^{\rm l} (k_{\rm v} \hat{M}^{\rm s} + \hat{M}^{\rm l}) + \frac{10^6}{V_0}\frac{R_{\rm p}^2}{M_0 g} \left( \frac{\hat{\mu}_{\rm v}^{\rm g}}{\hat{\mu}_{\rm t}} \right) p (X_{\rm v}^{\rm l}) + \frac{10^6}{V_0} \frac{m_{\rm v}^{\rm e}}{M_0} + \frac{10^6}{V_0} \frac{m_{\rm v}^{\rm o}}{M_0} = \hat{X}^{\rm init} \hat{M}^{\rm m}
\end{equation}
\subsubsection{Partial Pressure}
Convert the partial pressure expression to non-dimensional form.  First, introduce $\hat{X}^{\rm l}$:
\begin{equation}
p( X_{\rm v}^{\rm l} ) = \left( \frac{X_{\rm v}^{\rm l}}{\alpha} \right)^\beta
 = \left( \frac{10^6 V_0 X_{\rm v}^{\rm l}}{10^6 V_0 \alpha} \right)^\beta
 \end{equation}
 Therefore:
 \begin{equation}
p( \hat{X}^{\rm l} ) = \left( \frac{V_0 \hat{X}^{\rm l}}{10^6 \alpha} \right)^\beta
\end{equation}
Second, introduce the non-dimensional Henry's constant:
\begin{equation}
\hat{\alpha} = \frac{10^6 \alpha}{V_0} P_0^\frac{1}{\beta}
\end{equation}
In the code, we input $\alpha$ in units of ppm/Pa$^{1/\beta}$.  So to non-dimensionalise the input we just need to divide by $V_0$ and multiply by $P_0^{1/\beta}$, as shown above (i.e., the factor of $10^6$ is include in the definition of ppm).  Therefore, the non-dimensional partial pressure is:
\begin{equation}
\hat{p} ( \hat{X}^{\rm l} ) = \left( \frac{\hat{X}^{\rm l}}{\hat{\alpha}} \right) ^ {\beta}, \qquad \frac{d \hat{p}}{d \hat{X}^{\rm l}} = \frac{\beta}{\hat{\alpha}} \left( \frac{\hat{X}^{\rm l}}{\hat{\alpha}} \right)^{\beta-1}
\end{equation}
Now the mass balance becomes, noting that a factor of $P_0$ appears in the numerator on the penultimate term on the LHS because we introduce the non-dimensional pressure:
\begin{equation}
\hat{X}^{\rm l} (k_{\rm v} \hat{M}^{\rm s} + \hat{M}^{\rm l}) + \frac{10^6}{V_0} \frac{R_{\rm p}^2 P_0}{M_0 g} \left( \frac{\hat{\mu}_{\rm v}^{\rm g}}{\hat{\mu}_{\rm t}} \right) \hat{p} (\hat{X}^{\rm l}) + \frac{10^6}{V_0} \frac{m_{\rm v}^{\rm e}}{M_0} = \hat{X}^{\rm init} \hat{M}^{\rm m}
\end{equation}
Note that $m_{\rm v}^{\rm e}$ excludes a factor of $4\pi$ since we are only considering scaled masses now.
\subsubsection{Gravity and Radius}
Gravity is non-dimensionalised in SPIDER as:
\begin{equation}
g = \frac{S_0 T_0}{R_0} \hat{g}
\end{equation}
and the radius of the planet $R_p$ is non-dimensionalised by $R_0$.  Hence the mass balance becomes:
\begin{equation}
\hat{X}^{\rm l} (k_{\rm v} \hat{M}^{\rm s} + \hat{M}^{\rm l}) + \frac{10^6}{V_0}\frac{\hat{R}_{\rm p}^2 P_0 R_0^3}{M_0 S_0 T_0 \hat{g}} \left( \frac{\hat{\mu}_{\rm v}^{\rm g}}{\hat{\mu}_{\rm t}} \right) \hat{p} (\hat{X}^{\rm l}) + \frac{10^6}{V_0} \frac{m_{\rm v}^{\rm e}}{M_0} = \hat{X}^{\rm init} \hat{M}^{\rm m}
\end{equation}
Now the scaling constants in the penultimate term on the LHS are, according to the non-dimensional scheme in SPIDER:
\begin{equation}
\frac{P_0 R_0^3}{M_0 S_0 T_0} = \frac{\rho_0 S_0 T_0 R_0^3}{\rho_0 R_0^3 S_0 T_0} = 1
\end{equation}
\subsubsection{Non-dimensional volatile mass balance}
Therefore the mass balance in non-dimensional form is, \myemph{where again, masses of the solid, liquid, mantle, and escape reservoir, are scaled masses without the $4 \pi$ term}:
\boxedeq{}{\hat{X}^{\rm l} (k_{\rm v} \hat{M}^{\rm s} + \hat{M}^{\rm l}) + \frac{10^6}{V_0} \frac{\hat{R}_{\rm p}^2}{\hat{g}} \left( \frac{\hat{\mu}_{\rm v}^{\rm g}}{\hat{\mu}_{\rm t}} \right) \hat{p} (\hat{X}^{\rm l}) + \frac{10^6}{V_0} \frac{m_{\rm v}^{\rm e}}{M_0} + \frac{10^6}{V_0} \frac{m_{\rm v}^{\rm o}}{M_0} = \hat{X}^{\rm init} \hat{M}^{\rm m} \label{eq:volevo}}
Compared to the dimensional form (Eq.~\ref{eq:dimvolatile}), there is an extra scaling factor in front of the atmosphere (and escape) terms, to take account of the fact that we have converted mass fraction quantities to scaled ppm.  This extra scaling factor cannot be absorbed by a $\hat{X}^{\rm l}$ term since the volatile concentration is wrapped up inside the partial pressure formula.  Therefore, the (scaled) non-dimensional masses of volatiles are defined as:
\begin{equation}
\hat{m}_{\rm v}^{\rm s} = \hat{X}^{\rm l} k_{\rm v} \hat{M}^{\rm s}
\end{equation}
\begin{equation}
\hat{m}_{\rm v}^{\rm l} = \hat{X}^{\rm l} \hat{M}^{\rm l}
\end{equation}
\begin{equation}
\hat{m}_{\rm v}^{\rm g} = \frac{10^6}{V_0} \frac{\hat{R}_{\rm p}^2}{\hat{g}} \left( \frac{\hat{\mu}_{\rm v}^{\rm g}}{\hat{\mu}_{\rm t}} \right) \hat{p} (\hat{X}^{\rm l})
\end{equation}
\begin{equation}
\hat{m}_{\rm v}^{\rm e} = \frac{10^6}{V_0} \frac{m_{\rm v}^{\rm e}}{M_0}
\end{equation}
\begin{equation}
\hat{m}_{\rm v}^{\rm o} = \frac{10^6}{V_0} \frac{m_{\rm v}^{\rm o}}{M_0}
\end{equation}
But remember that the reservoir of escaped volatile is not defined per se, but rather the rate at which volatiles escape (see subsequent sections).
%%%%
\subsubsection{Dimensionalising}
In SPIDER we provide a dimensional scaling for each parameter to give a meaningful output for plotting and analysis.  For the \myemph{physical mantle reservoir masses of melt and solid phases (note $4 \pi$ term)}:
\begin{equation}
M^{\rm sp} = \hat{M}^{\rm s} \cdot 4 \pi M_0, \qquad M^{\rm lp} = \hat{M}^{\rm l} \cdot 4 \pi M_0
\end{equation}
Now for each volatile, we compute the dimensional mass in the liquid, solid, and gas phase, as well as the escaped reservoir:
\begin{equation}
m_{\rm v}^{\rm sp} = \hat{m}_{\rm v}^{\rm s} \cdot 4 \pi M_0 \cdot \left( \frac{V_0}{10^6} \right)
\end{equation}
\begin{equation}
m_{\rm v}^{\rm lp} = \hat{m}_{\rm v}^{\rm l} \cdot 4 \pi M_0 \cdot \left( \frac{V_0}{10^6} \right)
\end{equation}
\begin{equation}
m_{\rm v}^{\rm gp} = \hat{m}_{\rm v}^{\rm g} \cdot 4 \pi M_0 \cdot \left( \frac{V_0}{10^6} \right)
\end{equation}
\begin{equation}
m_{\rm v}^{\rm ep} = \hat{m}_{\rm v}^{\rm ep} \cdot 4 \pi M_0 \cdot \left( \frac{V_0}{10^6} \right)
\end{equation}
\begin{equation}
m_{\rm v}^{\rm o} = \hat{m}_{\rm v}^{\rm o} \cdot 4 \pi M_0 \cdot \left( \frac{V_0}{10^6} \right)
\end{equation}
%%%%
\subsection{Initial Volatile Concentration}
For an initial condition it is desirable to prescribe $\hat{X}^{\rm init}$, but we must then compute $\hat{X}^{\rm l}$ according to the mass balance since this is the quantity that is actually solved for (i.e., evolved with time).  We assume the magma ocean is completely molten at $t=0$, i.e. all the mantle mass is liquid, and no volatiles have yet escaped.  Therefore:
\begin{equation}
\hat{M}^{\rm s} = \hat{m}_{\rm v}^{\rm e} = \hat{m}_{\rm v}^{\rm o} = 0, \qquad \hat{M}^{\rm l} = \hat{M}^{\rm m}
\end{equation}
So the mass balance for the initial condition is:
\begin{equation}
\hat{X}^{\rm l} \hat{M}^{\rm m} + \frac{10^6}{V_0} \frac{\hat{R}_{\rm p}^2}{\hat{g}} \left( \frac{\hat{\mu}_{\rm v}^{\rm g}}{\hat{\mu}_{\rm t}} \right)\hat{p} (\hat{X}^{\rm l}) = \hat{X}^{\rm init} \hat{M}^{\rm m}
\end{equation}
Substitute in the expression for the partial pressure:
\begin{equation}
\hat{X}^{\rm l} \hat{M}^{\rm m} + \frac{10^6}{V_0} \frac{\hat{R}_{\rm p}^2}{\hat{g}} \left( \frac{\hat{\mu}_{\rm v}^{\rm g}}{\hat{\mu}_{\rm t}} \right)\left( \frac{\hat{X}^{\rm l}}{\hat{\alpha}} \right)^\beta = \hat{X}^{\rm init} \hat{M}^{\rm m}
\end{equation}
Divide through by the (non-dimensional) mantle mass $\hat{M}^m$:
\begin{equation}
\hat{X}^{\rm l} + \frac{10^6}{V_0} \frac{\hat{R}_{\rm p}^2}{\hat{M}^{\rm m} \hat{g}} \left( \frac{\hat{\mu}_{\rm v}^{\rm g}}{\hat{\mu}_{\rm t}} \right)\left( \frac{\hat{X}^{\rm l}}{\hat{\alpha}} \right)^\beta = \hat{X}^{\rm init}
\end{equation}
where the mean molar mass of the atmosphere is given by Eq.~\ref{eq:atmosphere_molar_mass} and recall that the partial pressures must sum to give the total pressure (Eq.~\ref{eq:atmosphere_molar_mass}).  For the initial condition, we want to specify the \myemph{total initial abundance of a given volatile, relative to the mantle mass}.  This means that we know the RHS and must solve for $\hat{X}^l$, since the code integrates the liquid mantle abundance.  Furthermore, since every volatile relation relies on the others through the mean molar mass of the atmosphere, we have a coupled system of $p$ non-linear equations that we must solve to determine the initial liquid abundance of every volatile species.
%%%%
%%%%
%%%%
\subsection{Evolution Equation}
Remember that we compute a RHS in SPIDER that represents the time-derivative (update) of a solution quantity.  We are going to solve the volatile evolution equations within the system of equations that include the update to entropy, etc.  \textbf{Strictly speaking, it is not necessary to integrate to find the equilibrium volatile abundance in the melt in the simplest situations.  You could save computations (and accumulated error) by instead solving Eq.~\ref{eq:volevo}, but this would not allow you to introduce time-dependent atmospheric escape for example.}  Therefore, taking the time derivative of the volatile mass balance equation:
\begin{equation}
\frac{d \hat{X}^{\rm l}}{d t} (k_{\rm v} \hat{M}^{\rm s} + \hat{M}^{\rm l}) + \hat{X}^{\rm l} \left( k_{\rm v} \frac{d \hat{M}^{\rm s}}{d t} + \frac{d \hat{M}^{\rm l}}{d t} \right) + \frac{10^6}{V_0} \frac{\hat{R}_{\rm p}^2 \hat{\mu}_{\rm v}^{\rm g}}{\hat{g}} \frac{d}{dt} \left( \frac{\hat{p}(\hat{X}^{\rm l})}{\hat{\mu}_{\rm t}} \right) + \frac{10^6}{V_0 M_0} \frac{dm_{\rm v}^{\rm e}}{dt} + \frac{10^6}{V_0 M_0} \frac{dm_{\rm v}^{\rm o}}{dt} = 0
\end{equation}
\subsubsection{Mean molar mass of atmosphere}
The mean molar mass $\hat{\mu}_{\rm t}$ of the atmosphere evolves during outgassing and therefore is a time-dependent quantity.  By the product rule:
\begin{equation}
\frac{d}{dt} \left( \frac{\hat{p}(\hat{X}^{\rm l})}{\hat{\mu}_{\rm t}} \right) = \hat{p}(\hat{X}^{\rm l}) \frac{d}{dt} \left( \frac{1}{\hat{\mu}_{\rm t}} \right) + \frac{1}{\hat{\mu}_{\rm t}} \frac{d}{dt} \left( \hat{p}(\hat{X}^{\rm l}) \right)
\label{eq:atmos_evo}
\end{equation}
The second term can be calculated using the chain rule:
\begin{equation}
\frac{1}{\hat{\mu}_{\rm t}} \frac{d}{dt} \left( \hat{p}(\hat{X}^{\rm l}) \right) = \frac{1}{\hat{\mu}_{\rm t}} \left( \frac{d \hat{p}}{d \hat{X}^{\rm l}} \frac{d \hat{X}^{\rm l}}{d t} \right)
\end{equation}
%%% two species only %%%
\subsubsection{Two volatile species}
To deal with the first term, recall that for two species (Eq.~\ref{eq:atmosphere_molar_mass}):
\begin{equation}
\hat{\mu}_{\rm t} = \left( \frac{\hat{p}_C}{\hat{p}_C+\hat{p}_H} \right) \hat{\mu}_C + \left( \frac{\hat{p}_H}{\hat{p}_C+\hat{p}_H} \right) \hat{\mu}_H
\end{equation}1
where subscript $C$ denotes CO$_2$ and $H$ denotes H$_2$O, and $\hat{p}$ is partial pressure and $\hat{\mu}$ is molar mass.  Taking the time derivative of $\hat{\mu}_{\rm t}$ noting that only the $\mu$'s are independent of time:
\begin{equation}
\frac{d}{dt} \left( \frac{1}{\hat{\mu}_{\rm t}} \right) = -\hat{\mu}_{\rm t}^{-2} \frac{d}{dt} \hat{\mu}_{\rm t}
\end{equation}
\begin{equation}
\frac{d}{dt} \hat{\mu}_{\rm t} = \frac{d}{dt} \left[ \left( \frac{\hat{p}_C}{\hat{p}_C+\hat{p}_H} \right) \hat{\mu}_C + \left( \frac{\hat{p}_H}{\hat{p}_C+\hat{p}_H} \right) \hat{\mu}_H \right]
\end{equation}
This is tedious but straightforward to compute (and Mathematica helps!):
\begin{align}
\frac{d}{dt} \hat{\mu}_{\rm t} &= \left( \frac{\hat{\mu}_C}{\hat{p}_C+\hat{p}_H} \right) \frac{d\hat{p}_C}{dt} - \frac{\hat{\mu}_C \hat{p}_C}{(\hat{p}_C+\hat{p}_H)^2} \left( \frac{d\hat{p}_C}{dt}+\frac{d\hat{p}_H}{dt} \right)\\
&+ \left( \frac{\hat{\mu}_H}{\hat{p}_C+\hat{p}_H} \right) \frac{d\hat{p}_H}{dt} - \frac{\hat{\mu}_H \hat{p}_H}{(\hat{p}_C+\hat{p}_H)^2} \left( \frac{d\hat{p}_C}{dt}+\frac{d\hat{p}_H}{dt} \right)
\end{align}
Simplifying:
\begin{equation}
\frac{d}{dt} \hat{\mu}_{\rm t} = \frac{(\hat{\mu}_C-\hat{\mu}_H)}{(\hat{p}_C+\hat{p}_H)^2} \left( \hat{p}_H \frac{d\hat{p}_C}{dt} - \hat{p}_C \frac{d\hat{p}_H}{dt} \right)
\end{equation}
Putting it all together (i.e., returning to Eq.~\ref{eq:atmos_evo}):
\begin{equation}
\frac{d}{dt} \left( \frac{\hat{p}(\hat{X}^{\rm l})}{\hat{\mu}_{\rm t}} \right) = -\frac{\hat{p}(\hat{X}^{\rm l})}{\hat{\mu}_{\rm t}^2} \frac{(\hat{\mu}_C-\hat{\mu}_H)}{(\hat{p}_C+\hat{p}_H)^2} \left( \hat{p}_H \frac{d\hat{p}_C}{dt} - \hat{p}_C \frac{d\hat{p}_H}{dt} \right) + \frac{1}{\hat{\mu}_{\rm t}} \left( \frac{d \hat{p}}{d \hat{X}^{\rm l}} \frac{d \hat{X}^{\rm l}}{d t} \right)
\end{equation}
To recap, we are considering the concentration of a given volatile $\hat{X}^{\rm l}$, which can either be CO$_2$ or H$_2$O, and we need to know the time derivative of this quantity.  \myemph{Through the mean molar mass of the atmosphere, the volatiles are now coupled, and we can no longer determine the time derivative of either volatile independently of the other.  Rather we must solve a coupled system of equations at each time step to give us the time update of all the volatiles under consideration.}.  The above equations may be expressed slightly differently, but should be identical to the equations in \citet[Appendix A,][]{BKW19}.
\subsubsection{$n$ volatile species}
For $n$ species, where $(t)$ is used to emphasise the time-dependent quantities, we just need to derive a general expression for the rate-of-change of the mean molar mass of the atmosphere:
\begin{equation}
\hat{\mu}_{\rm t}(t) = \sum_i^n \left( \frac{\hat{p}_i(t) \hat{\mu}_i}{\sum_j^n \hat{p}_j(t)} \right)
\end{equation}
Time derivative:
\begin{equation}
\frac{d \hat{\mu}_{\rm t}}{d t} = \sum_i^n \hat{\mu}_i \frac{d}{d t} \left( \frac{\hat{p}_i(t)}{\sum_j^n \hat{p}_j(t)} \right)
\end{equation}
Note that the term within the time derivative is the volume mixing ratio of each volatile species.  Break apart the derivative using the product rule:
\begin{equation}
\frac{d \hat{\mu}_{\rm t}}{d t} = \sum_i^n \hat{\mu}_i \left[ \hat{p}_i \frac{d}{d t} \left( \frac{1}{\sum_j^n \hat{p}_j} \right) + \frac{1}{\sum_j^n \hat{p}_j} \frac{d \hat{p}_i}{d t} \right]
\end{equation}
Evaluate:
\begin{equation}
\frac{d \hat{\mu}_{\rm t}}{d t} = \sum_i^n \hat{\mu}_i \left[ \frac{-\hat{p}_i}{\left( \sum_j^n \hat{p}_j \right)^2} \sum_j^n \frac{d \hat{p}_j}{d t} + \frac{1}{\sum_j^n \hat{p}_j} \frac{d \hat{p}_i}{d t} \right]
\end{equation}
Looks simpler when you substitute in the total surface pressure $P_s$ and break apart some terms to show the symmetry (and confirm that the units are correct):
\begin{equation}
\frac{d \hat{\mu}_{\rm t}}{d t} = \sum_i^n \hat{\mu}_i \left[ \frac{1}{P_s} \frac{d \hat{p}_i}{d t} - \frac{\hat{p}_i}{P_s} \frac{1}{P_s} \frac{d P_s}{d t} \right]
\end{equation}
Putting it all together:
\begin{equation}
\frac{d}{dt} \left( \frac{\hat{p}(\hat{X}^{\rm l})}{\hat{\mu}_{\rm t}} \right) = -\frac{\hat{p}(\hat{X}^{\rm l})}{\hat{\mu}_{\rm t}^2} \sum_i^n \hat{\mu}_i \left[ \frac{1}{P_s} \frac{d \hat{p}_i}{d t} - \frac{\hat{p}_i}{P_s} \frac{1}{P_s} \frac{d P_s}{d t} \right] + \frac{1}{\hat{\mu}_{\rm t}} \left( \frac{d \hat{p}}{d \hat{X}^{\rm l}} \frac{d \hat{X}^{\rm l}}{d t} \right)
\end{equation}
The above equations may be expressed slightly differently, but should be identical to the equations in \citet[Appendix A,][]{BKW19}.
%%%%
%%%%
\subsubsection{Returning to the evolution equation}
Recall that:
\begin{equation}
\hat{M}^{\rm s} + \hat{M}^{\rm l} = \hat{M}^{\rm m} \equiv \text{constant}
\label{eq:mantle_mass}
\end{equation}
Therefore, we can now just express in terms of the total mantle mass and the mass of liquid (melt) and its time derivative.  \myemph{The evolution equation is, therefore:}
\begin{align}
&\frac{d \hat{X}^{\rm l}}{d t} \left(k_{\rm v} \hat{M}^{\rm m} + (1-k_{\rm v}) \hat{M}^{\rm l} \right)
+ \hat{X}^{\rm l} (1-k_{\rm v}) \frac{d \hat{M}^{\rm l}}{d t} + \nonumber \\
&\quad \frac{10^6}{V_0} \frac{\hat{R}_{\rm p}^2 \hat{\mu}_{\rm v}^{\rm g}}{\hat{g}}
\left[
-\frac{\hat{p}(\hat{X}^{\rm l})}{\hat{\mu}_{\rm t}^2} \sum_i^n \hat{\mu}_i \left[ \frac{1}{P_s} \frac{\partial \hat{p}_i}{\partial t} - \frac{\hat{p}_i}{P_s} \frac{1}{P_s} \frac{\partial P_s}{\partial t} \right] + \frac{1}{\hat{\mu}_{\rm t}} \left( \frac{d \hat{p}}{d \hat{X}^{\rm l}} \frac{d \hat{X}^{\rm l}}{d t} \right)
\right]+ \nonumber \\
& \qquad \frac{10^6}{V_0 M_0} \frac{dm_{\rm v}^{\rm e}}{dt} + \frac{10^6}{V_0 M_0} \frac{dm_{\rm v}^{\rm o}}{dt} = 0
\end{align}
This cannot be separated into the form of the time derivative equal to a RHS, so the time derivative must instead be numerically calculated within the time stepper.
%%%%
%%%%
%%%%
\subsubsection{Atmospheric escape formulation}
The expression for escape due to surface heating is \citep{JOY15}: % Eq. 2b
\begin{equation}
\frac{d m_{\rm v}^{\rm esc}}{dt} = \left( \frac{d m_{\rm v}^{\rm g}}{dt} \right) \mathcal{R} (1 + \lambda_s) \exp(-\lambda_s) 
\end{equation}
where $\mathcal{R}$ is a fitting parameter based on molecular kinetic simulations of N$_2$ \citep{VJT11,VTE11} and $\lambda_s$ is the surface value of the Jean's parameter $\lambda_s$.  For us, $\lambda_s$ and $\mathcal{R}$ may just be treated as a constant scaling for simplicity.  From dimensional analysis we can see that the terms following the mass derivative must be non-dimensional constants.
\begin{equation}
\lambda_s =  \frac{G M_p^{\rm p} \mu_{\rm v}}{R_p k_b T_s N_A} = \frac{g R_p \mu_{\rm v}}{k_b T_s N_A}
\end{equation}
where $G$ is the gravitational constant, $g$ surface gravity, $M_p$ mass of the planet (superscript p denotes physical mass), $\mu_v$ the molar mass of the volatile, $N_A$ Avogadro's number (units per mol), $k_b$ Boltzmann constant, and $T_s$ surface temperature.  Remember that we deal with scaled masses in SPIDER, but since the Jeans parameter is a physical quantity, we must introduce the correct physical scaling.  Now in fact, the physical $g$ is treated as a constant input parameter by SPIDER and therefore absorbs the factor of $4 \pi$ that is included in the planetary mass.  Non-dimensionalising:
\begin{equation}
\lambda_s = \left( \frac{S_0 T_0 R_0 M_0}{R_0 S_0 \rho_0 R_0^3 T_0} \right) \frac{\hat{g} \hat{R}_p \hat{\mu}_{\rm v}}{\hat{k}_b \hat{T}_s N_A} = \frac{\hat{g} \hat{R}_p \hat{\mu}_{\rm v}^{\rm g}}{\hat{k}_b \hat{T}_s N_A}
\end{equation}
We already computed the growth rate of the atmosphere, i.e. the source rate, and therefore:
\begin{align}
\frac{d \hat{m}_{\rm v}^{\rm esc}}{dt} &= \frac{10^6}{V_0} \frac{\hat{R}_{\rm p}^2 \hat{\mu}_{\rm v}^{\rm g}}{\hat{g}}
\left[
-\frac{\hat{p}(\hat{X}^{\rm l})}{\hat{\mu}_{\rm t}^2} \sum_i^n \hat{\mu}_i \left[ \frac{1}{P_s} \frac{d \hat{p}_i}{d t} - \frac{\hat{p}_i}{P_s} \frac{1}{P_s} \frac{d P_s}{d t} \right] + \frac{1}{\hat{\mu}_{\rm t}} \left( \frac{d \hat{p}}{d \hat{X}^{\rm l}} \frac{d \hat{X}^{\rm l}}{d t} \right)
\right]\\
& \times \mathcal{R} (1 + \lambda_s) \exp(-\lambda_s) 
\end{align}
It is now obvious that we can scale the source time in the evolution equation by a factor to account for atmospheric escape:
\begin{equation}
\mathcal{F}_e = 1+\mathcal{R} (1 + \lambda_s) \exp(-\lambda_s)
\end{equation}
%%%%
%%%%
\subsubsection{Global mass of liquid and solid}
The current version of SPIDER computes the hydrostatic pressure profile \emph{a priori} based on the Adams-Williamson equation of state.  This means that for simplicity we can compute the liquid mass by multiplying the mass of a particular radial shell $d\hat{m}_i$, with some function $g$, and then sum:% the melt fraction $\phi$, and then sum:
\begin{equation}
\hat{M}^{\rm l} = \sum_i^N g_i \ d\hat{m}_i, \qquad \hat{M}^{\rm s} = \sum_i^N (1-g_i) \ d\hat{m}_i
%\hat{M}^{\rm l} = \sum_i^N \phi_i \ d\hat{m}_i, \qquad \hat{M}^{\rm s} = \sum_i^N (1-\phi_i) \ d\hat{m}_i
\end{equation}
Note that $d\hat{m}_i$ is constant and computed from the Adam-Williamson equation of state.  \myemph{Formally, there is an inconsistency, since the lookup tables determine the density of the liquid and solid phases, but these will not necessarily give a constant mantle mass for all time!  This is why I choose to use the hydrostatic pressure profile instead, because by construction this will ensure the total mantle mass is unchanging during the course of a model run.}.  Also note that by construction Eq.~\ref{eq:mantle_mass} holds.
The derivative of liquid mass with respect to time is given by the chain rule:
\begin{equation}
\frac{d \hat{M}^{\rm l}}{dt} = \sum_i^N \frac{d g_i}{d S_i} \frac{dS_i}{dt} dm_i, \qquad \text{ for } 0 < \phi < 1
\end{equation}
The change in solid mass can be similarly computed, and since we are dealing with only two phases:
\begin{equation}
\frac{d \hat{M}^{\rm l}}{dt} = -\frac{d \hat{M}^{\rm s}}{dt}
\end{equation}
In the code, I experimented with smoothing across the liquidus and solidus for these integrated mass quantities, but early testing suggested that no smoothing performed better. See around line 400 in twophase.c.
%%%%
%%%%
\subsubsection{Batch crystallisation}
If we assume that the magma ocean batch crystallises, then:
\begin{equation}
g_i(t) = \phi_i(t)
\end{equation}
Basically, the ability of a partial of mass to retain volatiles is only dependent on its melt fraction.  Also:
\begin{equation}
\frac{d \hat{M}^{\rm l}}{dt} = \sum_i^N \frac{d \phi_i}{d S_i} \frac{dS_i}{dt} dm_i = \sum_i^N \frac{1}{\Delta S_{{\rm fus}i}} \frac{dS_i}{dt} dm_i, \qquad \text{ for } 0 < \phi < 1
\end{equation}
%%%%
%%%%
\subsubsection{Fractional crystallisation}
We can retain the same basic formulation as for batch crystallisation, but now add a weighting factor that penalises the ability of low melt fractions to store volatiles:
\begin{equation}
g_i(t) = \frac{\phi(t)}{2} \left(1+ \tanh\left(\frac{\phi(t)-\phi_c}{\phi_w} \right)\right)
\end{equation}
where $\phi(t)$ is melt fraction, $\phi_c$ the critical melt fraction, and $\phi_w$ a smoothing width.  Compute the time derivative:
\begin{equation}
\frac{d g_i}{d S_i} =\frac{1}{\Delta S_{i,fus}} \frac{1}{2\phi_w} \frac{1}{\cosh^2{x_i}}, \qquad x_i = \frac{\phi_i(t)-\phi_c}{\phi_w}
\end{equation}
%%%%
%%%%
%%%%
\subsection{Grey atmosphere model}
Given the mass of volatiles in the atmosphere, we can compute an effective emissivity of the atmosphere using a grey atmosphere model.  The model is described in detail in \cite{AM85} in the appendix, and the key equations are presented more clearly in \cite{ET08}.  Note also that \cite{AND10} describes the model in Sect. 3.7.2 in his book, but his formulation is slightly different from \cite{AM85} as he considers the boundary condition at the top of the atmosphere in terms of long-wave only.  \cite{AND10} considers the \myemph{net upward long wave irradiance}:
\begin{equation}
F_z = F_{\uparrow}^l - F_{\downarrow}^l = const, \qquad F_{\downarrow}^l=0, \qquad F_\uparrow^l(0)=F_0, \qquad \implies F_z=F_0
\end{equation}
Whereas \cite{AM85} consider the \myemph{net upward flux}, which includes both short and long wave components:
\begin{equation}
F_{atm} = F_{\uparrow} - F_{\downarrow} = const, \qquad F_{\downarrow}=F_\infty, \qquad \implies F_{\uparrow} = F_{atm} + F_\infty
\end{equation}
The incoming short wave irradiation is $F_\infty$ in \cite{AM85} and $F_0$ in \cite{AND10}.  And $F_{\uparrow}^s=0$ since reflection is not considered (by either author).

\subsubsection{Optical depth}
For a given volatile we compute the optical depth (and remember that optical depth is a non-dimensional quantity) \dbnote{TODO: clean up code with absolute value of gravity, since gravity is negative in SPIDER?} \citep[Eq. A18,][]{AM85}:
\begin{equation}
\tau^\ast = \frac{3 \kappa^\prime p(\tau^\ast)}{2g}
%\tau^\ast = \frac{3 m_{\rm v}^{\rm g}}{8 \pi R_{\rm p} ^2} \sqrt{\frac{k_{\rm abs} g}{3 p_0}} = \frac{3p}{2g} \sqrt{\frac{k_{\rm abs} g}{3 p_0}} = \frac{3 p k_{\rm abs}^\prime}{2g}
\label{eq:tau}
\end{equation}
Remember that Henry's law gives us the \myemph{surface partial pressure}, so we can use this directly to compute the \myemph{surface optical depth}.  We don't need to go via the atmospheric mass of the volatile, but if we wanted we could also use:
\begin{equation}
\tau^\ast_s = \frac{3 \kappa^\prime m_{\rm v}^{\rm g}}{8 \pi R_p^2} \left( \frac{\mu_{\rm t}}{\mu_{\rm v}^{\rm g}} \right)
\end{equation}
Again, the molar mass ratio does not seem to appear in other author's formulations.  Non-dimensionalising:
\begin{equation}
\tau^\ast_s = \frac{ 3 \hat{m}_{\rm v}^{\rm g} \cdot 4 \pi M_0}{8 \pi \hat{R}_p^2 R_0^2} \frac{V_0}{10^6} \left( \frac{\hat{\mu}_{\rm v}^{\rm g}}{\hat{\mu}_{\rm t}} \right) \left[ \frac{\hat{k}_{\rm abs} R_0^2 \hat{g} S_0 T_0}{3 \hat{p}_0 P_0 M_0 R_0} \right]^{\frac{1}{2}}
\end{equation}
And since (reassuringly) all the dimensional scalings cancel:
\boxedeq{}{\tau^\ast_s = \frac{3}{2} \frac{V_0}{10^6} \frac{\hat{m}_{\rm v}^{\rm g}}{\hat{R}_{\rm p}^2} \left( \frac{\hat{\mu}_{\rm v}^{\rm g}}{\hat{\mu}_{\rm t}} \right) \sqrt{\frac{\hat{k}_{\rm abs} \hat{g}}{3 \hat{p}_0}}}
\subsubsection{Effective emissivity}
Finally, the optical depths for each volatile $\tau_j$  are combined to give an effective emissivity, which can then be fed into the usual formulation for a grey-body \citep[Eq. A14,][]{AM85}:
\begin{equation}
\epsilon = \frac{2}{\sum_j \tau_j^\ast +2}
\end{equation}
\subsubsection{Atmosphere structure (1-D)}
The temperature profile, as a function of optical depth, is \citep[Eq.~A15,][]{AM85}:
\begin{equation}
T(\tau^\ast) = \left( T_0^4 \frac{(\tau^\ast+1)}{2} +T_\infty^4 \right)^\frac{1}{4}
\label{eq:Ttau}
\end{equation}
where:
\begin{equation}
T_0 = \left( \frac{F_{atm}}{\sigma} \right)^\frac{1}{4}
\end{equation}
This only depends on the (scaled) optical depth, which is an effective optical depth obtained by addition of the optical depth of each volatile.

\subsubsection{Optical depth to height above planetary surface}
\begin{equation}
\rho = \frac{n \mu}{V}
\end{equation}
where $n$ is number of moles, $\mu$ molar mass, and $V$ volume. Ideal gas law:
\begin{equation}
pV = nRT \implies \rho = \frac{p \mu}{R_g T}
\end{equation}
Hydrostatic equilibrium:
\begin{equation}
\frac{dp}{dz} = - \rho g \implies \frac{dp}{p} = -\frac{\mu g}{R_g T}dz
\end{equation}
If you assume that $T$ is constant (or nearly constant), then the above is trivial to integrate.  But we also have a 1-D temperature of profile that we would like to honour, so we have to do more work.  Using Eq.~\ref{eq:tau}:
\begin{equation}
\frac{dp}{p} = \frac{d \tau^\ast}{\tau^\ast} \implies \frac{d \tau^\ast}{\tau^\ast} = -\frac{\mu g}{R_g T}dz
\end{equation}
We also know $T(\tau^\ast)$ from Eq.~\ref{eq:Ttau}, and therefore:
\begin{equation}
\frac{dz}{d\tau^\ast} = \frac{1}{\tau^\ast \beta} \left( T_0^4 \frac{(\tau^\ast+1)}{2} +T_\infty^4 \right)^\frac{1}{4}, \qquad \text{where } \beta = -\frac{\mu g}{R_g}
\label{eq:dzdtau}
\end{equation}
Everything on the RHS (except $\tau$) is known.  The \myemph{combined optical depth} at the surface is known from Eq.~\ref{eq:tau}, which gives an initial value:
\begin{equation}
\tau^\ast (z=0) = \tau^\ast_s
\label{eq:dzdtau2}
\end{equation}
We can take $\mu$ as the mean molecular weight of the atmosphere, i.e. molecular weights of H$_2$O and CO$_2$, weighted by the mixing ratio of each species in the atmosphere.  We can now solve for $z(\tau^\ast)$ using Eqs.~\ref{eq:dzdtau} and \ref{eq:dzdtau2}, and thus provide a mapping from the (aggregate scaled) optical depth to the vertical height coordinate above the planetary surface.  We (PS and I) attempted to solve the equations analytically using Mathematica, but we have thus far been unable to obtain a Real valued function only.  But testing with Mathematica and Python reveal that the equation is trivial to integrate numerically, so I instead implemented a RK4 algorithm (actually, Simpson's rule, since the ODE is only a function of $\tau$) in SPIDER.
\subsubsection{Flux and $T_{eqm}$}
\dbnote{TODO: implement this (and check equations)}
The formulation in SPIDER involves specifying (or calculating) an effective temperature $T_{eqm}$, but this relates to the incoming stellar flux:
\begin{equation}
F_{sun} = \sigma T_{eqm}^4 = (1-\alpha) \frac{F_0^\prime}{D^2}
\end{equation}
where $\alpha$ is the bolometric albedo (usually around 0.2), $F_0^\prime$ is the averaged solar constant over the surface, and $D$ the planet--star distance (AU).
\begin{equation}
F_0^\prime = \frac{F_0}{4}
\end{equation}
where $F_0$ is the solar constant:
\begin{equation}
F_{0,t} = F_{0,t=0}^\prime \left[ 1 + 0.4 \left( 1 - \frac{t}{t_0} \right) \right] ^ {-1}
\end{equation}